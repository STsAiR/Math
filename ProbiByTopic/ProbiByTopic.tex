\documentclass[12pt]{article}
\usepackage{amsthm,amssymb,amsfonts,amsmath,amstext,systeme,graphicx,float,tabularx}
\marginparwidth 0pt
\oddsidemargin -1.2 truecm
\evensidemargin  0pt 
\marginparsep 0pt
\topmargin -2.2truecm
\linespread{1}
\textheight 25.8 truecm
\textwidth 18.5 truecm
\newenvironment{remark}{\noindent{\bf Remark }}{\vspace{0mm}}
\newenvironment{remarks}{\noindent{\bf Remarks }}{\vspace{0mm}}
\newenvironment{question}{\noindent{\bf Question }}{\vspace{0mm}}
\newenvironment{questions}{\noindent{\bf Questions }}{\vspace{0mm}}
\newenvironment{note}{\noindent{\bf Note }}{\vspace{0mm}}
\newenvironment{summary}{\noindent{\bf Summary }}{\vspace{0mm}}
\newenvironment{back}{\noindent{\bf Background}}{\vspace{0mm}}
\newenvironment{conclude}{\noindent{\bf Conclusion}}{\vspace{0mm}}
\newenvironment{concludes}{\noindent{\bf Conclusions}}{\vspace{0mm}}
\newenvironment{dill}{\noindent{\bf Description of Dill's model}}{\vspace{0mm}}
\newenvironment{maths}{\noindent{\bf Mathematics needed}}{\vspace{0mm}}
\newenvironment{inst}{\noindent{\bf Instructions}}{\vspace{0mm}}
\newenvironment{notes}{\noindent{\bf Notes }}{\vspace{0mm}}
\newenvironment{theorem}{\noindent{\bf Theorem }}{\vspace{0mm}}
\newenvironment{example}{\noindent{\bf Example }}{\vspace{0mm}}
\newenvironment{examples}{\noindent{\bf Examples }}{\vspace{0mm}}
\newenvironment{topics}{\noindent{\bf Topics}}{\vspace{0mm}}
\newenvironment{outcomes}{\noindent{\bf Expected Learning Outcomes}}{\vspace{0mm}}
\newenvironment{lemma}{\noindent{\bf Lemma }}{\vspace{0mm}}
\newenvironment{solution}{\noindent{\it Solution}}{\vspace{2mm}}
\newcommand{\ds}{\displaystyle}
\newcommand{\un}{\underline}
\newcommand{\bs}{\boldsymbol}

\begin{document}

\baselineskip 18 pt
\begin{center}
	{\large \bf Probability By Topic}
\end{center}
\vspace{0.05cm}

\begin{enumerate}
	\item \textbf{HKDSE MATH CORE 2019 Past Paper I Q15}\\
	There are 21 boys and 11 girls in a class. If 5 students are selected from the class to form a committee consisting of at least 1 boy, how many different committees can be formed?	 \\(3 marks)

    \item \textbf{HKDSE MATH CORE 2020 Past Paper I Q15}\\
	In a box, there are 3 blue plates, 7 green plates and 9 purple plates. If 4 plates are randomly selected from the box at the same time, find
	\begin{enumerate}
		\item[(a)] the probability that 4 plates of the same colour are selected; \\(3 marks)
		\item[(b)] the probability that at least 2 plates of different clours are selected. \\(2 marks)
	\end{enumerate}


    \item \textbf{HKDSE MATH CORE 2021 Past Paper I Q15}\\
	A queue is randomly formed by 7 teachers and 3 students.
	\begin{enumerate}
		\item[(a)] How many different queues can be formed? \\(1 marks)
		\item[(b)] Find the probability that no students are next to each other in the queue. \\(3 marks)
	\end{enumerate}


	\item \textbf{HKDSE MATH CORE 2022 Past Paper I Q15}\\
	There are 10 boys and 12 girls in a class. If 4 students are randomly selected from the class to form a committee.
	\begin{enumerate}
		\item[(a)] find the probability that there are 2 boys and 2 girls in the committee. \\(2 marks)
		\item[(b)] find the probability that the number of boys and the number of girls in the committee are different. \\(2 marks)
	\end{enumerate}




	\item \textbf{HKDSE MATH CORE 2023 Past Paper I Q15}\\
	In a box, there are 4 red balls and 4 black balls. From the box, 2 balls are randomly chosen at the same time.
	\begin{enumerate}
		\item[(a)] Find the probability that the 2 balls chosen are red. \\(2 marks)
		\item[(b)] In a bag, there are 8 red balls. The 2 balls form the box are put into the bag and then 3 balls are randomly chosen at the smae time from the bag. Find the probability that the 3 balls chosen are of the same colour. \\(2 marks)
	\end{enumerate}
	\item \textbf{HKDSE MATH CORE 2018 Past Paper I Q15}\\
	An eight-digit phone number is formed by a permutation of 2, 3, 4, 5, 6, 7, 8 and 9.
	\begin{enumerate}
		\item[(a)] How many different eight-digit phone numbers can be formed? \\(1 mark)
		\item[(b)] If the first digit and the last digit of an eight-digit phone number are odd numbers, how many different eight-digit phone numbers can be formed? \\(2 marks)
	\end{enumerate}
    \item \textbf{HKDSE MATH CORE 2017 Past Paper I Q17}\\
	In a bag, there are 4 green pens, 7 blue pens and 8 black pens. If 5 pens are randomly drawn from the bag at the same time,
	\begin{enumerate}
		\item[(a)] find the probability that exactly 4 green pens are drawn; \\(2 marks)
		\item[(b)] find the probability that exactly 3 green pens are drawn; \\(2 marks)
		\item[(c)] find the probability that not more than 2 green pens are drawn. \\(2 marks)
	\end{enumerate}
\end{enumerate}








\end{document}