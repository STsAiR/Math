\documentclass[12pt]{article}
\usepackage{amsthm,amssymb,amsfonts,amsmath,amstext,systeme}
\usepackage{graphicx,float}
\usepackage{tabularx}

\marginparwidth 0pt
\oddsidemargin -1.2 truecm
\evensidemargin  0pt 
\marginparsep 0pt
\topmargin -2.2truecm
\linespread{1}
\textheight 25.8 truecm
\textwidth 18.5 truecm
\newenvironment{remark}{\noindent{\bf Remark }}{\vspace{0mm}}
\newenvironment{remarks}{\noindent{\bf Remarks }}{\vspace{0mm}}
\newenvironment{question}{\noindent{\bf Question }}{\vspace{0mm}}
\newenvironment{questions}{\noindent{\bf Questions }}{\vspace{0mm}}
\newenvironment{note}{\noindent{\bf Note }}{\vspace{0mm}}
\newenvironment{summary}{\noindent{\bf Summary }}{\vspace{0mm}}
\newenvironment{back}{\noindent{\bf Background}}{\vspace{0mm}}
\newenvironment{conclude}{\noindent{\bf Conclusion}}{\vspace{0mm}}
\newenvironment{concludes}{\noindent{\bf Conclusions}}{\vspace{0mm}}
\newenvironment{dill}{\noindent{\bf Description of Dill's model}}{\vspace{0mm}}
\newenvironment{maths}{\noindent{\bf Mathematics needed}}{\vspace{0mm}}
\newenvironment{inst}{\noindent{\bf Instructions}}{\vspace{0mm}}
\newenvironment{notes}{\noindent{\bf Notes }}{\vspace{0mm}}
\newenvironment{theorem}{\noindent{\bf Theorem }}{\vspace{0mm}}
\newenvironment{example}{\noindent{\bf Example }}{\vspace{0mm}}
\newenvironment{examples}{\noindent{\bf Examples }}{\vspace{0mm}}
\newenvironment{topics}{\noindent{\bf Topics}}{\vspace{0mm}}
\newenvironment{outcomes}{\noindent{\bf Expected Learning Outcomes}}{\vspace{0mm}}
\newenvironment{lemma}{\noindent{\bf Lemma }}{\vspace{0mm}}
\newenvironment{solution}{\noindent{\it Solution}}{\vspace{2mm}}
\newcommand{\ds}{\displaystyle}
\newcommand{\un}{\underline}
\newcommand{\bs}{\boldsymbol}

\begin{document}

\baselineskip 18 pt
\begin{center}
	{\large \bf HKDSE MATH CORE 2024 Past Paper I}\\
	\vspace{2 mm}

\end{center}
\vspace{0.05cm}

\begin{enumerate}
	\item \textbf{HKDSE MATH CORE 2024 Past Paper I Q1}\\
	Simplify $\dfrac{2}{4h - 7} - \dfrac{3}{6h - 5}$. \\(3 marks)

	\item \textbf{HKDSE MATH CORE 2024 Past Paper I Q2}\\
	Make $x$ the subject of the formula $\dfrac{Ax + C}{B} = 3x$. \\(3 marks)	

	\item \textbf{HKDSE MATH CORE 2024 Past Paper I Q3}\\
	Factorize
	\begin{enumerate}
		\item[(a)] $6r^2 - 13rs - 28s^2$,
		\item[(b)] $4r - 14s + 6r^2 - 13rs - 28s^2$.
	\end{enumerate}
	(3 marks)

	\item \textbf{HKDSE MATH CORE 2024 Past Paper I Q4}\\
	\begin{enumerate}
		\item[(a)] Find the range of values of $x$ which satisfy both $\dfrac{5x + 7}{4} - 1 < 2x$ and $3x + 9 \geq 0$.
		\item[(b)] Write down the least integer satisfying both inequalities in (a).
	\end{enumerate}
	(4 marks)

	\item \textbf{HKDSE MATH CORE 2024 Past Paper I Q5}\\
	Let $a$, $b$ and $c$ be non-zero numbers such that $5a = 6c$ and $\dfrac{2b + 7c}{b + c} = 4$. Find $\dfrac{5a + 8b}{2b + 3c}$. \\(4 marks)


	\item \textbf{HKDSE MATH CORE 2024 Past Paper I Q6}\\
    The marked price of a calculator is 40\% higher than its cost. The calculator is sold at a discount of 25\% on its marked price and the profit is $\$13$. 
    Find the marked price of the calculator. \\(4 marks)

	\item \textbf{HKDSE MATH CORE 2024 Past Paper I Q7}\\
    In a polar coordinate system, $O$ is the pole. The polar coordinates of the points $P$, $Q$ and $R$ are $(11, 59^\circ)$, $(60, 149^\circ)$ and $(144, 239^\circ)$ respectively.
	\begin{enumerate}
		\item[(a)] Find $\angle POQ$
		\item[(b)] Are $P$, $O$ and $R$ collinear? Explain your answer.
		\item[(c)] Find the perimeter of $\triangle PQR$.
	\end{enumerate}
	(4 marks)
	
	\item \textbf{HKDSE MATH CORE 2024 Past Paper I Q8}\\
    In Figure 1, $E$ is the point of intersection of $AC$ and $BD$. It is given that $\angle ACB = \angle ADB = 90^\circ$ and $AD = BC$.
	\begin{figure}[H]
		\centering
		\includegraphics[width = .5\linewidth]{2024Figure1.1}
		% \caption*{}
	\end{figure}
	\begin{enumerate}
		\item[(a)] Prove that $\triangle ABC \cong \triangle BAD$.
		\item[(b)] If $AD = 12$ cm and $DE = 9$ cm, find the pentagon of $ABCED$.
	\end{enumerate}
	(5 marks)

	\item \textbf{HKDSE MATH CORE 2024 Past Paper I Q9}\\
	The stem-and-leaf diagram below shoes the distribution of the numbers of working hours of a group of workers in a week.
	\begin{table}[htbp]
		\centering
        \begin{tabular}{|c|c|c|c|c|c|c|}
			\hline
			Number of keys & 3 & 4 & 5 & 6 & 7 & 8 \\ \hline
			Number of housewives & 10 & 9 & 4 & 3 & 4 & $k$
			\\
			\hline
			\end{tabular}
		\label{tab:addlabel}
	\end{table}
    If a housewife is randomly selected from the group, then the probability that she owns more than 6 keys is $\dfrac{5}{18}$.
	\begin{enumerate}
		\item[(a)] Find $k$.
		\item[(b)] Write down the mean, mode and median of the distribution.
	\end{enumerate}
	(5 marks)

	\item \textbf{HKDSE MATH CORE 2024 Past Paper I Q10}\\
	It is given that $A$ and $B$ are two distinct points in a rectangular coordinate plane. Let $P$ be a moving point in the rectangular coordinate plane such that $P$ is equidistant from $A$ and $B$. Denote the locus of $P$ by $\Gamma$.
	\begin{enumerate}
		\item[(a)] Describe the geometric relationship between $\Gamma$ and $AB$. \\(1 marks)
		\item[(b)] Suppose that the coordinates of $A$ are $(2, -4)$ and the equation of $\Gamma$ is $3x + y - 12 = 0$. Find 
		\begin{enumerate}
			\item[(i)] the equation of the straight line which passes through $A$ and $B$,
			\item[(ii)] the equation of the circle with $AB$ as a diameter.
		\end{enumerate}
		(5 marks)
	\end{enumerate}

	\item \textbf{HKDSE MATH CORE 2024 Past Paper I Q11}\\
	The table below shoes the distribution of the numbers of calculators owned by a class of students.
	$$\begin{array}{|c|c|c|c|c|}
		\hline
		\text{Number of calculators owned} & 1 & 2 & 3 & 4 \\
		\hline
		\text{Number of student} & 8 & 5 & n & 1 \\
		\hline
	\end{array}$$
	The mean of the distribution is 2.
	\begin{enumerate}
		\item[(a)] Find the median, inter-quatile range and the variance of the above distribution. \\(5 marks)
		\item[(b)] Two students now withdraw from the class. It is found that the mean of the distribution remains unchanged. Is there any change in the range of the distribution due to the withdrawal of the two students? Explain your answer. \\(2 marks)
	\end{enumerate}

	\item \textbf{HKDSE MATH CORE 2024 Past Paper I Q12}\\
	It is given that $f(x)$ is partly constant and partly varies as $x^2$. Suppose that $f(10) = 62$ and $f(15) = 122$.
	\begin{enumerate}
		\item[(a)] Find $f(5)$. \\(3 marks)
		\item[(b)] Suppose that $U(0, u)$ and $V(5,v)$ are points lying on the graph of $y = f(x)$. The horizontal line passing through $V$ cuts the $y$-axis at the point $W$. Denote the circle which passes through $U$, $V$ and $W$ by $C$. Express the circumference of $C$ in terms of $\pi$. \\(4 marks)
	\end{enumerate}

	\item \textbf{HKDSE MATH CORE 2024 Past Paper I Q13}\\
	Define $g(x) = x^3 + 5x^2 - 12x - 1$. Let $h(x) = 3x^4 + ax^3 - 16x^2 + bx + c$, where $a$, $b$ and $c$ are constants. When $h(x)$ is divided by $g(x)$, the quotient and the remainder are equal.
	\begin{enumerate}
		\item[(a)] Find the quotient when $h(x)$ is divided by $g(x)$. \\(3 marks)
		\item[(b)] How many rational roots does the equation $h(x) = 0$ have? Explain your answer. \\(4 marks)
	\end{enumerate}

	\item \textbf{HKDSE MATH CORE 2024 Past Paper I Q14}\\
	The base radius and the curved area of a solid metal reight circular cone are 14 cm and $700\pi$ cm$^2$ respectively.
	\begin{enumerate}
		\item[(a)] Find the height of the circular cone. \\(3 marks)
		\item[(b)] The circular cone is divided into a right circular cone $X$ anda frustum $Y$ by a plane which is parallel to its base. The curve surface area of $Y$ is 15 times the curved surface area of $X$.
		\begin{enumerate}
			\item[(i)] Express the volume of $Y$ in terms of $\pi$.
			\item[(ii)] If $Y$ is melted and recast into 2 identical solid spheres, find the diameter of each sphere.
		\end{enumerate}
		(5 marks)
	\end{enumerate}

	\item \textbf{HKDSE MATH CORE 2024 Past Paper I Q15}\\
	In a box, there are 4 red balls and 4 black balls. From the box, 2 balls are randomly chosen at the same time.
	\begin{enumerate}
		\item[(a)] Find the probability that the 2 balls chosen are red. \\(2 marks)
		\item[(b)] In a bag, there are 8 red balls. The 2 balls form the box are put into the bag and then 3 balls are randomly chosen at the smae time from the bag. Find the probability that the 3 balls chosen are of the same colour. \\(2 marks)
	\end{enumerate}

	\item \textbf{HKDSE MATH CORE 2024 Past Paper I Q16}
	\begin{enumerate}
		\item[(a)] Let $a$ and $b$ be real constants. If the roots of the equation $x^2 + ax + b = 0$ are $p$ and $5p$, prove that $5a^2 = 36b$. \\(2 marks)
		\item[(b)] Denote the circle $x^2 + y^2 - 6x - 12y + 20 = 0$ by $C$. Find the constant $m$ such that the straight line $y = mx$ cuts $C$ at the points $Q$ and $R$ with $OQ:QR = 1:4$, where $O$ is the origin. \\(3 marks)
	\end{enumerate}

	\item \textbf{HKDSE MATH CORE 2024 Past Paper I Q17}
	\begin{enumerate}
		\item[(a)] It is given that $WXY$ is a triangle, where $WX = 6$ cm , $XY = 5$ cm and $\angle WYX = 70^\circ$. Find $\angle XWY$. \\(2 marks)
		\item[(b)] Figure 3 shows the pyramid $WXYZ$, where $WZ = XZ = YZ$. The base of this pyramid is the triangle $WXY$ described in (a).
		% \begin{figure}[H]
		% 	\centering
		% 	\includegraphics[width = .3\linewidth]{2024Figure1.3}
		% \end{figure}
		It is given that the angle between $WZ$ and the triangle $WXY$ is $30^\circ$. Does the angle between the triangles $WXY$ and $XYZ$ exceeed $45^\circ$? Explain your answer. \\(4 marks)
	\end{enumerate}

	\item \textbf{HKDSE MATH CORE 2024 Past Paper I Q18}\\
	Suppose that $\alpha , 7 , \beta $ is a geometric sequence, where $q < \alpha < \beta$.
	\begin{enumerate}
		\item[(a)] Express $\log_7{\alpha}$ in terms of $\log_7{\beta}$. \\(3 marks)
		\item[(b)] If $\log_{\beta}{\alpha}, \log_{7}{\beta}, \log_{\alpha}{\beta}$ is an arithmetic sequence, find the common difference of the arithmetic sequence. \\(5 marks)
	\end{enumerate}

	\item \textbf{HKDSE MATH CORE 2024 Past Paper I Q19}\\
	The coordinates of the points $P$ and $Q$ are $(50 , 0)$ and $(32 , t)$ respectively, where $t > 0$. Denote the origin by $O$. Let $R$ be a point such that $OQ$ is a median of $\triangle OPR$. Suppose that $G$ and $H$ are the circumcentre and the orthocentre of $\triangle OPR$ respectively.
	\begin{enumerate}
		\item[(a)] Express the coordinates of $G$ and $H$ in terms of $t$. \\(5 marks)
		\item[(b)] Let $S$ be a point lying on $OP$ such that $QS$ is perpendicular to $OP$. It is given that $\angle PQS = \angle POQ$.
		\begin{enumerate}
			\item[(i)] By considering $\tan{\angle PQS}$, prove that $t = 24$.
			\item[(ii)] Are $O$, $G$ and $Q$ collinear? Explain your answer.
			\item[(iii)] Denote the in-centre of $\triangle OPR$ by $I$. Find the ratio of the area of $\triangle GHR$ to the area of $\triangle IPQ$.
		\end{enumerate}
		(7 marks)
	\end{enumerate}


\end{enumerate}
\end{document}