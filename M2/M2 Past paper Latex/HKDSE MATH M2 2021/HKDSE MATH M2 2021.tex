\documentclass[12pt]{article}
\usepackage{amsthm,amssymb,amsfonts,amsmath,amstext,systeme}

\marginparwidth 0pt
\oddsidemargin -1.2 truecm
\evensidemargin  0pt 
\marginparsep 0pt
\topmargin -2.2truecm
\linespread{1}
\textheight 25.8 truecm
\textwidth 18.5 truecm
\newenvironment{remark}{\noindent{\bf Remark }}{\vspace{0mm}}
\newenvironment{remarks}{\noindent{\bf Remarks }}{\vspace{0mm}}
\newenvironment{question}{\noindent{\bf Question }}{\vspace{0mm}}
\newenvironment{questions}{\noindent{\bf Questions }}{\vspace{0mm}}
\newenvironment{note}{\noindent{\bf Note }}{\vspace{0mm}}
\newenvironment{summary}{\noindent{\bf Summary }}{\vspace{0mm}}
\newenvironment{back}{\noindent{\bf Background}}{\vspace{0mm}}
\newenvironment{conclude}{\noindent{\bf Conclusion}}{\vspace{0mm}}
\newenvironment{concludes}{\noindent{\bf Conclusions}}{\vspace{0mm}}
\newenvironment{dill}{\noindent{\bf Description of Dill's model}}{\vspace{0mm}}
\newenvironment{maths}{\noindent{\bf Mathematics needed}}{\vspace{0mm}}
\newenvironment{inst}{\noindent{\bf Instructions}}{\vspace{0mm}}
\newenvironment{notes}{\noindent{\bf Notes }}{\vspace{0mm}}
\newenvironment{theorem}{\noindent{\bf Theorem }}{\vspace{0mm}}
\newenvironment{example}{\noindent{\bf Example }}{\vspace{0mm}}
\newenvironment{examples}{\noindent{\bf Examples }}{\vspace{0mm}}
\newenvironment{topics}{\noindent{\bf Topics}}{\vspace{0mm}}
\newenvironment{outcomes}{\noindent{\bf Expected Learning Outcomes}}{\vspace{0mm}}
\newenvironment{lemma}{\noindent{\bf Lemma }}{\vspace{0mm}}
\newenvironment{solution}{\noindent{\it Solution}}{\vspace{2mm}}
\newcommand{\ds}{\displaystyle}
\newcommand{\un}{\underline}
\newcommand{\bs}{\boldsymbol}

\begin{document}

\baselineskip 18 pt
\begin{center}
	{\large \bf HKDSE MATH M2 2021}\\
	\vspace{2 mm}

\end{center}
\vspace{0.05cm}

\begin{enumerate}
	\item \textbf{HKDSE Math M2 2021 Q1}\\
	Let $\displaystyle f(x) = \frac{1}{3x^{2}+4}$. Find $f'(x)$ from first principles. \\
	(4 marks)


	\item \textbf{HKDSE Math M2 2021 Q2}\\
	Using mathematical induction, prove that $\displaystyle \sum_{k = 1}^{n} (3k^{5} + k^{3}) = \frac{n^{3} (n+1)^{3}}{2}$ for all positive integers $n$. \\
	(5 marks)


	\item \textbf{HKDSE Math M2 2021 Q3}\\
	The coefficient of $x^2$ in the expansion of $(1-4x)^n$ is 240, where $n$ is a positive integer. Find
	\begin{enumerate}
		\item[(a)]
		$n$,
		\item[(b)]
		the coefficient of $x^4$ in the expansion of $\displaystyle(1-4x)^n\left(1+\frac{2}{x}\right)^5$. 
	\end{enumerate}
	(6 marks)

	\item \textbf{HKDSE Math M2 2021 Q4}
	\begin{enumerate}
		\item [(a)] Prove that $\cos{2x} + \cos {4x} + \cos {6x} = 4\cos{x}\cos{2x}\cos{3x} -1$.
		\item [(b)] Solve the equation $\cos{4\theta} + \cos{8\theta} + \cos{12\theta} = -1$, where $\displaystyle0 \leq \theta \leq \frac{\pi}{2}$.
	\end{enumerate}
	(6 marks)

	\item \textbf{HKDSE Math M2 2021 Q5}\\
	Define $\displaystyle r(x) = \frac{x^3 - x^2 -2x + 3}{(x-1)^2}$ for all real numbers $x \neq 1$.
	\begin{enumerate}
		\item [(a)] Find the asymptote(s) of the graph of $y = r(x)$.
		\item [(b)] Find $\displaystyle\frac{d}{dx}r(x)$.
		\item [(c)] Someone claims that there is only one point of inflexion of the graph of $y = r(x)$. Do you agree? Explain your answer.
	\end{enumerate}
	(7 marks)

	\item \textbf{HKDSE Math M2 2021 Q6}\\
	Consider the curve $\Gamma : y = e^{2x-6}$. Denote the normal to $\Gamma$ at the point $(3,1)$ by $L$. Let $c$ be the $x$-intercept of $L$. Find
	\begin{enumerate}
		\item [(a)]$c$;
		\item [(b)]the area of the region bounded by $L$, $\Gamma$ and the straight line $x=c$.
	\end{enumerate}
	(7 marks)

	\item \textbf{HKDSE Math M2 2021 Q7}
	\begin{enumerate}
		\item [(a)]Using integration by parts, find $\displaystyle\int (\ln{x})^2 \,dx$.
		\item [(b)] Consider the curve $\displaystyle C : y = \sqrt{x}\ln{(x^2+1)}$, where $x\geq 0$. Let $R$ be the region bounded by $C$, the straight line $x=1 $ and the $x$-axis. Find the volume of the solid of revolution generated by revolving R about the $x$-axis.
	\end{enumerate}
	(7 marks)

	\item \textbf{HKDSE Math M2 2021 Q8}\\
	Consider the system of linear equations in real variables $x$, $y$, $z$
		$$(E) : \left\{\begin{matrix}
		x&  +&(d-1)y&  +&(d+3)z& = &4-d  \\
		2x&  +&(d+2)y&  -&z& = & 2d-5 \\
		3x&  +&(d+4)y&  +&5z& = & 2 \\
		\end{matrix}\right. \text{,  where } d \in \mathbb{R} .$$
		It is given that $(E)$ has infinitely many solutions.
	\begin{enumerate}
		\item [(a)] Find $d$. Hence, solve $(E)$.
		\item [(b)] Someone claims that $(E)$ has a real solution $(x,y,z)$ satisfying $xy + 2xz = 3$. Is the claim correct? Explain your answer. 
	\end{enumerate}
	(8 marks)

	\item \textbf{HKDSE Math M2 2021 Q9}
	\begin{enumerate}
		\item [(a)] Let $\displaystyle \frac{-\pi}{2} < \theta < \frac{\pi}{2}$.
		\begin{enumerate}
			\item [(i)] Find $\displaystyle \frac{d}{d\theta} \ln{(\sec{\theta} + \tan{\theta})}$.
			\item [(ii)] Using the result of (a)(i), find $\displaystyle \int \sec{\theta} \,d\theta$. Hence, find $\displaystyle \int \sec^3{\theta}\,d\theta$. 
		\end{enumerate}
		(4 marks)
		\item [(b)] Let $g(x)$ and $h(x)$ be continuous functions defined on $\mathbb{R}$ such that $g(x)+g(-x) =1$ and $h(x)=h(-x) $ for all $x \in \mathbb{R}$. \\Using integration by substitution, prove that $\displaystyle\int_{-a}^{a}g(x)h(x)\,dx = \int_{0}^{a}h(x)\,dx$ for any $a \in \mathbb{R}$.\\(3 marks)
		\item Evaluate $\displaystyle \int_{-1}^{1} \frac{3^x x^2}{(3^x+3^{-x})\sqrt{x^2+1}}\,dx$.\\(5 marks)
	\end{enumerate}


	\item \textbf{HKDSE Math M2 2021 Q10}\\
	Denote the graph of $y = \sqrt{x^2+36} $ and the graph of $y = - \sqrt{(20-x)^2+16}$ by $F$ and $G$ respectively, where $0 < x<20$. Let $P$ be a moving point on $F$. The vertical line passing through $P$ cuts $G$ at the point $Q$. Denote the $x$-coordinate of $P$ by $u$. It is given that the length of $PQ$ attains its minimum value when $u=a$.
	\begin{enumerate}
		\item [(a)] Find $a$.\\(4 marks)
		\item [(b)] The horizontal line passing through $P$ cuts the $y$-axis at the point $R$ while the horizontal line passing through $Q$ cuts the $y$-axis at the point $S$.
		\begin{enumerate}
			\item [(i)] Someone claims that the area of the rectangle $PQSR$ attains its minimum value when $u = a$. Do you agree? Explain your answer. 
			\item [(ii)] The length of $OP$ increases at a constant rate of 28 units per minute. Find the rate of change of the perimeter of the rectangle $PQSR$ when $u = a$.
		\end{enumerate}
		(9 marks)
	\end{enumerate}


	\item \textbf{HKDSE Math M2 2021 Q11}\\
	Define $P = \begin{pmatrix}
		\sin{\theta}&\cos{\theta}\\
		-\cos{\theta}&\sin{\theta}\\
		\end{pmatrix}$, where $\displaystyle \frac{\pi}{2} < \theta < \pi$.
	\begin{enumerate}
		\item [(a)] Let $A = 
			\begin{pmatrix}
			\alpha&\beta\\
			\beta&-\alpha\\
			\end{pmatrix}$, where $\alpha, \beta \in \mathbb{R}$.\\
			Prove that $PAP^{-1} = \begin{pmatrix}
			-\alpha \cos{2\theta}+\beta \sin{2\theta} &-\beta\cos{2\theta}-\alpha \sin{2\theta}\\
			-\beta\cos{2\theta}-\alpha\sin{2\theta}&\alpha\cos{2\theta}-\beta \sin{2\theta}\\
			\end{pmatrix}$. \\(3 marks)
		\item[(b)]Let $B = \begin{pmatrix}
			1&\sqrt{3}\\
			\sqrt{3}&-1\\
			\end{pmatrix}$.
		\begin{enumerate}
			\item [(i)] Find $\theta$ such that $PBP^{-1} = \begin{pmatrix}
				\lambda&0\\
				0&\mu\\
				\end{pmatrix}$, where $\lambda$, $\mu \in \mathbb{R}$.
			\item [(ii)]Using the result of (b)(i), prove that $B^n = 2^{n-2} \begin{pmatrix}
				(-1)^n+3&\sqrt{3}(-1)^{n+1} + \sqrt{3}\\
				\sqrt{3}(-1)^{n+1}+\sqrt{3}&3(-1)^n+1\\
				\end{pmatrix}$
				for any positive integer $n$.
			\item [(iii)] Evaluate $(B^{-1})^{555}$. 
		\end{enumerate}
		(9 marks)
	\end{enumerate}

	\item \textbf{HKDSE Math M2 2021 Q12}\\
	The position vectors of the points $A$, $B$, $C$ and $D$ are  $t\textbf{i} +14 \textbf{j}+s \textbf {k}$, $12\textbf{i} -s \textbf{j}-2 \textbf {k}$, $(s+2)\textbf{i} -16 \textbf{j}+10 \textbf {k}$ and $-t\textbf{i} +(s+2) \textbf{j}+14 \textbf {k}$  respectively, where $s$, $t \in \mathbb{R}$. Suppose that $\overrightarrow{AB}$ is parallel to $5\textbf{i} -4 \textbf{j}-2 \textbf {k}$. \\Denote the plane which contains $A$, $B$ and $C$ by $\Pi$.
	\begin{enumerate}
		\item [(a)]Find
		\begin{enumerate}
			\item [(i)]$s$ and $t$,
			\item [(ii)]the area of $\triangle ABC$,
			\item [(iii)]the volume of the tetrahedron $ABCD$,
			\item [(iv)]the shortest distance from $D$ to $\Pi$.
		\end{enumerate}
		(9 marks)
		\item [(b)]Let $E$ be the projection of  $D$ on $\Pi$. Is $E$ the circumcentre of $\triangle ABC$? Explain your answer.\\(4 marks)
	\end{enumerate}
\end{enumerate}
\end{document}