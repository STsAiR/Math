\documentclass[12pt]{article}
\usepackage{amsthm,amssymb,amsfonts,amsmath,amstext,systeme}

\marginparwidth 0pt
\oddsidemargin -1.2 truecm
\evensidemargin  0pt 
\marginparsep 0pt
\topmargin -2.2truecm
\linespread{1}
\textheight 25.8 truecm
\textwidth 18.5 truecm
\newenvironment{remark}{\noindent{\bf Remark }}{\vspace{0mm}}
\newenvironment{remarks}{\noindent{\bf Remarks }}{\vspace{0mm}}
\newenvironment{question}{\noindent{\bf Question }}{\vspace{0mm}}
\newenvironment{questions}{\noindent{\bf Questions }}{\vspace{0mm}}
\newenvironment{note}{\noindent{\bf Note }}{\vspace{0mm}}
\newenvironment{summary}{\noindent{\bf Summary }}{\vspace{0mm}}
\newenvironment{back}{\noindent{\bf Background}}{\vspace{0mm}}
\newenvironment{conclude}{\noindent{\bf Conclusion}}{\vspace{0mm}}
\newenvironment{concludes}{\noindent{\bf Conclusions}}{\vspace{0mm}}
\newenvironment{dill}{\noindent{\bf Description of Dill's model}}{\vspace{0mm}}
\newenvironment{maths}{\noindent{\bf Mathematics needed}}{\vspace{0mm}}
\newenvironment{inst}{\noindent{\bf Instructions}}{\vspace{0mm}}
\newenvironment{notes}{\noindent{\bf Notes }}{\vspace{0mm}}
\newenvironment{theorem}{\noindent{\bf Theorem }}{\vspace{0mm}}
\newenvironment{example}{\noindent{\bf Example }}{\vspace{0mm}}
\newenvironment{examples}{\noindent{\bf Examples }}{\vspace{0mm}}
\newenvironment{topics}{\noindent{\bf Topics}}{\vspace{0mm}}
\newenvironment{outcomes}{\noindent{\bf Expected Learning Outcomes}}{\vspace{0mm}}
\newenvironment{lemma}{\noindent{\bf Lemma }}{\vspace{0mm}}
\newenvironment{solution}{\noindent{\it Solution}}{\vspace{2mm}}
\newcommand{\ds}{\displaystyle}
\newcommand{\un}{\underline}
\newcommand{\bs}{\boldsymbol}

\begin{document}

\baselineskip 18 pt
\begin{center}
	{\large \bf HKDSE MATH M2 2016}\\
	\vspace{2 mm}

\end{center}
\vspace{0.05cm}

\begin{enumerate}
	\item \textbf{HKDSE Math M2 2016 Q1}\\
	Expand $(5+x)^4$. Hence, find the constant term in the expansion of $(5+x)^4\left(1-\displaystyle\frac{2}{x}\right)^3$. \\(5 marks)

	\item \textbf{HKDSE Math M2 2016 Q2}\\
	Prove that $\displaystyle\frac{1}{\sqrt{x}} - \frac{1}{\sqrt{x+h}} = \frac{h}{(x+h)\sqrt{x} + x\sqrt{x+h}}$. Hence, find $\displaystyle \frac{d}{dx} \sqrt{\displaystyle\frac{3}{x}}$ from first principles. \\(5 marks)

	\item \textbf{HKDSE Math M2 2016 Q3}\\
	Consider the curve $C : y = 2e^x$, where $x>0$. It is given that $P$ is a point lying on $C$. The horizontal line which passes through $P$ cuts the $y$-axis at the point $Q$. Let $O$ be the origin. Denote the $x$-coordinate of $P$ by $u$. 
	\begin{enumerate}
		\item [(a)]Express the area of $\triangle OPQ$ in terms of $u$.
		\item [(b)]If $P$ moves along $C$ such that $OQ$ increases at a constant rate of $6$ units per second, find the rate of change of the area of $\triangle OPQ$ when $u=4$.
	\end{enumerate}
	(5 marks)

	\item \textbf{HKDSE Math M2 2016 Q4}\\
	Define $f(x) = \displaystyle\frac{2x^2 + x + 1}{x - 1}$ for all $x \neq 1$. Denote the graph of $y = f(x)$ by $G$. Find 
	\begin{enumerate}
		\item [(a)]the asymptote(s) of $G$,
		\item [(b)]The slope of the normal to $G$ at the point $(2,11)$.
	\end{enumerate}
	(7 marks)

	\item \textbf{HKDSE Math M2 2016 Q5}
	\begin{enumerate}
		\item [(a)]Using mathematical induction, prove that $\displaystyle\sum_{k=1}^{n} (-1)^k k^2 = \frac{(-1)^n n(n+1)}{2}$ for all positive integers $n$. 
		\item [(b)]Using (a), evaluate $\displaystyle\sum_{k=3}^{333} (-1)^{k+1} k^2$.
	\end{enumerate}
	(6 marks)

	\item \textbf{HKDSE Math M2 2016 Q6}
	\begin{enumerate}
		\item [(a)]Prove that $x+1$ is a factor of $4x^3+2x^2-3x-1$. 
		\item [(b)]Express $\cos{3\theta}$ in terms of $\cos{\theta}$.
		\item [(C)]Using the results of (a) and (b), prove that $\cos{\displaystyle\frac{3\pi}{5}} = \displaystyle\frac{1-\sqrt{5}}{4}$. \\(6 marks)
	\end{enumerate}
	(6 marks)

	\item \textbf{HKDSE Math M2 2016 Q7}
	\begin{enumerate}
		\item [(a)]Using integration by substitution, find $\displaystyle\int (1+\sqrt{t+1})^2 \,dt$. 
		\item [(b)]Consider the curve $\Gamma : y = 4x^2 - 4x$, where $1 \leq x \leq 4$. Let $R$ be the region bounded by $\Gamma$, the straight line $y=48$ and the two axes. Find the volume of the solid of revolution generated by revolving $R$ about the $y$-axis.
	\end{enumerate}
	(8 marks)

	\item \textbf{HKDSE Math M2 2016 Q8}\\
	Let $n$ be a positive integer.
	\begin{enumerate}
		\item [(a)]Define $A = 
		\begin{pmatrix}
			1&0\\1&1\\
		\end{pmatrix}$. Evaluate 
		\begin{enumerate}
			\item [(i)]$A^2$, 
			\item [(ii)]$A^n$,
			\item [(iii)]$(A^{-1})^n$.
		\end{enumerate}
		\item [(b)]Evaluate
		\begin{enumerate}
			\item [(i)]$\displaystyle\sum_{k=0}^{n-1} 2^k$,
			\item [(ii)]$\begin{pmatrix}
				1&0\\1&2\\
			\end{pmatrix} ^n$.
		\end{enumerate}
	\end{enumerate}
	(8 marks)

	\item \textbf{HKDSE Math M2 2016 Q9}\\
	Let $a$ and $b$ be constants. Define $f(x) = x^3 + ax^2 + bx + 5$ for all real numbers $x$. Denote the curve $y = f(x)$ by $C$. It is given that $P(-1,10)$ is a turning point of $C$.
	\begin{enumerate}
		\item [(a)]Find $a$ and $b$.\\(3 marks)
		\item [(b)]Is $P$ a maximum point of $C$? Explain your answer. \\(2 marks)
		\item [(c)]Find the minimum value(s) of $f(x)$. \\(2 marks)
		\item [(d)]Find the point(s) of inflexion of $C$. \\(2 marks)
		\item [(e)]Let $L$ be the tangent to $C$ at $P$. Find the area of the region bounded by $C$ and $L$. \\(4 marks)
	\end{enumerate}

	\item \textbf{HKDSE Math M2 2016 Q10}
	\begin{enumerate}
		\item [(a)]Let $f(x)$ be a continuous function defined on the interval $[0,a]$, where $a$ is a positive constant. \\Prove that $\int_0^a f(x)\,dx = \int_0^a f(a-x)\,dx$. \\(3 marks)
		\item [(b)]Prove that $\displaystyle\int_{ 0}^{\frac{\pi}{4}} \ln{(1+\tan{x})}\,dx = \int_{0}^{\frac{\pi}{4}} \ln{\left(\frac{2}{1+\tan{x}}\right)}\,dx$. \\(3 marks)
		\item [(c)]Using (b), prove that $\displaystyle\int_{0}^{\frac{\pi}{4}} \ln {(1+\tan{x})}\,dx = \frac{\pi \ln 2}{8}$. \\(3 marks)
		\item [(d)]Using integration by parts, evaluate $\displaystyle\int_{ 0}^{\frac{\pi}{4}} \frac{x\sec^2{x}}{1+\tan{x}}\,dx$. \\(3 marks)
	\end{enumerate}

	\item \textbf{HKDSE Math M2 2016 Q11}
	\begin{enumerate}
		\item [(a)]Consider the system of linear equations in real variables $x$, $y$, $z$
		$$(E) : \left\{\begin{matrix}
		x &+&     y&-&      z&=&3  \\
		4x&+&    6y&+&     az&=&b  \\
		5x&+&(1-a)y&+&(3a-1)z&=&b-1\\
		\end{matrix}\right.\text{, where } a\text{ and } b\text{ are real numbers.}$$
		\begin{enumerate}
			\item [(i)]Assume that $(E)$ has a unique solution.
			\begin{enumerate}
				\item [(1)]Prove that $a\neq -2$ and $a \neq -12$.
				\item [(2)]Solve $(E)$. 
			\end{enumerate}
			\item [(ii)]Assume that $a = -2$ and $(E)$ is consistent.
			\begin{enumerate}
				\item [(1)]Find $b$. 
				\item [(2)]Solve $(E)$.
			\end{enumerate}
		\end{enumerate}
		(9 marks)
		\item [(b)] Is there a real solution of the system of linear equations
		$$\left\{\begin{matrix}
		x&+&y&-&z&=&3\\
		2x&+&3y&-&z&=&7\\
		5x&+&3y&-&7z&=&13\\
		\end{matrix}\right.$$
		satisfying $x^2+y^2-6z^2 > 14$? Explain your answer. \\(3 marks)
	\end{enumerate}

	\item \textbf{HKDSE Math M2 2016 Q12}\\
	Let $\overrightarrow{OA} = 2 \textbf{j} +2\textbf {k}$, 
		$\overrightarrow{OB} = 4\textbf{i} + \textbf{j} + \textbf {k}$ and 
		$\overrightarrow{OP} = \textbf{i} +t \textbf{j}$, where $t$ is a constant and $O$ is the origin. It is given that $P$ is equidistant from $A$ and $B$. 
	\begin{enumerate}
		\item [(a)]Find $t$.\\(3 marks)
		\item [(b)]Let $\overrightarrow{OC} = 2 \textbf{i} - \textbf{j} +4\textbf {k}$ and $\overrightarrow{OD} = 3 \textbf{i} +2 \textbf{j} +5\textbf {k}$. Denote the plane which contains $A$, $B$ and $C$ by $\Pi$.
		\begin{enumerate}
			\item [(i)]Find a unit vector which is perpendicular to $\Pi$. 
			\item [(ii)]Find the angle between $CD$ and  $\Pi$. 
			\item [(iii)]It is given that $E$ is a point lying on $\Pi$ such that 
			$\overrightarrow{DE}$ is perpendicular to $\Pi$. Let $F$ be a point such that $\overrightarrow{PF} = \overrightarrow{PA} + \overrightarrow{PB} + \overrightarrow{PC}$. Describe the geometric relationship between $D$, $E$ and $F$. Explain your answer.
		\end{enumerate}
		(10 marks)
	\end{enumerate}
\end{enumerate}
\end{document}