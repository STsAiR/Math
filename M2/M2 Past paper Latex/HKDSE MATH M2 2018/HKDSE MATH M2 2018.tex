\documentclass[12pt]{article}
\usepackage{amsthm,amssymb,amsfonts,amsmath,amstext,systeme}

\marginparwidth 0pt
\oddsidemargin -1.2 truecm
\evensidemargin  0pt 
\marginparsep 0pt
\topmargin -2.2truecm
\linespread{1}
\textheight 25.8 truecm
\textwidth 18.5 truecm
\newenvironment{remark}{\noindent{\bf Remark }}{\vspace{0mm}}
\newenvironment{remarks}{\noindent{\bf Remarks }}{\vspace{0mm}}
\newenvironment{question}{\noindent{\bf Question }}{\vspace{0mm}}
\newenvironment{questions}{\noindent{\bf Questions }}{\vspace{0mm}}
\newenvironment{note}{\noindent{\bf Note }}{\vspace{0mm}}
\newenvironment{summary}{\noindent{\bf Summary }}{\vspace{0mm}}
\newenvironment{back}{\noindent{\bf Background}}{\vspace{0mm}}
\newenvironment{conclude}{\noindent{\bf Conclusion}}{\vspace{0mm}}
\newenvironment{concludes}{\noindent{\bf Conclusions}}{\vspace{0mm}}
\newenvironment{dill}{\noindent{\bf Description of Dill's model}}{\vspace{0mm}}
\newenvironment{maths}{\noindent{\bf Mathematics needed}}{\vspace{0mm}}
\newenvironment{inst}{\noindent{\bf Instructions}}{\vspace{0mm}}
\newenvironment{notes}{\noindent{\bf Notes }}{\vspace{0mm}}
\newenvironment{theorem}{\noindent{\bf Theorem }}{\vspace{0mm}}
\newenvironment{example}{\noindent{\bf Example }}{\vspace{0mm}}
\newenvironment{examples}{\noindent{\bf Examples }}{\vspace{0mm}}
\newenvironment{topics}{\noindent{\bf Topics}}{\vspace{0mm}}
\newenvironment{outcomes}{\noindent{\bf Expected Learning Outcomes}}{\vspace{0mm}}
\newenvironment{lemma}{\noindent{\bf Lemma }}{\vspace{0mm}}
\newenvironment{solution}{\noindent{\it Solution}}{\vspace{2mm}}
\newcommand{\ds}{\displaystyle}
\newcommand{\un}{\underline}
\newcommand{\bs}{\boldsymbol}

\begin{document}

\baselineskip 18 pt
\begin{center}
	{\large \bf HKDSE MATH M2 2018}\\
	\vspace{2 mm}

\end{center}
\vspace{0.05cm}
\begin{enumerate}
	\item \textbf{HKDSE Math M2 2018 Q1}\\
	Let $\displaystyle f(x) = (x^2-1)e^x$.  Express $f(1+h)$ in terms of $h$. Hence, find  $f'(1)$ from first principles. \\(4 marks)

	\item  \textbf{HKDSE Math M2 2018 Q2}\\
	Expand $(x+3)^5$. Hence, find the coefficient of $x^3$ in the expansion of $(x+3)^5 \displaystyle\left(x - \frac{4}{x}\right) ^ 2$. \\(5 marks)

	\item \textbf{HKDSE Math M2 2018 Q3}
	\begin{enumerate}
		\item [(a)] If $\cot{A} = 3\cot{B}$, prove that $\sin{(A+B)} = 2\sin{(B-A)}$. 
		\item [(b)] Using (a), solve the equation $\displaystyle\cot{(x+\frac{4\pi}{9})} = 3\cot{(x+\frac{5\pi}{18})}$, where $0 \leq x \leq \displaystyle\frac{\pi}{2}$.
	\end{enumerate}
	(5 marks)


	\item \textbf{HKDSE Math M2 2018 Q4}
	\begin{enumerate}
		\item [(a)]Using integration by parts, find $\displaystyle\int u(5^u) \,du$. 
		\item [(b)]Define $f(x) = x(5^{2x})$ for all real numbers $x$. Find the area of the region bounded by the graph of $y = f(x)$, the straight line $x = 1$ and the $x$-axis.
	\end{enumerate}
	(6 marks)

	\item \textbf{HKDSE Math M2 2018 Q5}
	\begin{enumerate}
		\item [(a)]Using integration by substitution, find $\displaystyle\int x^3\sqrt{1 + x^2} \,dx$. 
		\item [(b)]At any point $(x,y)$ on the curve $\Gamma$, the slope of the tangent to $\Gamma$ is $15x^3\sqrt{1+x^2}$. The $y$-intercept of $\Gamma$ is 2. Find the equation of $\Gamma$.
	\end{enumerate}
	(7 marks)

	\item \textbf{HKDSE Math M2 2018 Q6}
	\begin{enumerate}
		\item [(a)]Using mathematical induction, prove that $\displaystyle \sum_{k = 1}^{n} k(k+4) = \frac{n(n+1)(2n+13)}{6}$ for all positive integers $n$.
		\item [(b)] Using (a), evaluate $\displaystyle \sum_{k = 333}^{555} \left(\frac{k}{112}\right)\left(\frac{k+4}{223}\right)$.
	\end{enumerate}
	(7 marks)

	\item \textbf{HKDSE Math M2 2018 Q7}\\
	Let $M = 
		\begin{pmatrix}
		7 &3 \\
		-1&5\\
		\end{pmatrix}$. Let $X = 
		\begin{pmatrix}
		a &6a \\
		b&c\\
		\end{pmatrix}$ be a non-zero real matrix such that $MX = XM$. 
	\begin{enumerate}
		\item[(a)]Express $b$ and $c$ in  terms of $a$. 
		\item[(b)]Prove that $X$ is a non-singular matrix.
		\item[(c)]Denote the transpose of $X$ by $X^T$. Express $(X^T)^{-1}$ in terms of $a$.
	\end{enumerate}
	(8 marks)

	\item \textbf{HKDSE Math M2 2018 Q8}\\
	Define $f(x) = \displaystyle \frac{A}{x^2-4x+7}$ for all real numbers $x$, where $A$ is a constant. It is given that the extreme value of $f(x)$ is 4. 
	\begin{enumerate}
		\item [(a)] Find $f'(x)$. 
		\item [(b)] Someone claims that there are at least two asymptotes of the graph of $ y =  f(x)$. Do you agree? Explain your answer. 
		\item [(c)] Find the point(s) of inflexion of the graph of $y =f(x)$.
	\end{enumerate}
	(8 marks)

	\item \textbf{HKDSE Math M2 2018 Q9}\\
	Consider the curve $C : y = \displaystyle\ln \sqrt{x}$, where $x > 1$. Let $P$ be a moving point lying on $C$. The normal to $C$ at $P$ cuts the $x$-axis at the point $Q$ while the vertical line passing through $P$ cuts the $x$-axis at the point $R$.  
	\begin{enumerate}
		\item [(a)]Denote the $x$-coordinate of $P$ by $r$. Prove that the $x$-coordinate of $Q$ is $\displaystyle\frac{4r^2+\ln r}{4r}$.  \\(3 marks)
		\item [(b)]Find the greatest area of $\triangle PQR$. \\(5 marks) 
		\item [(c)]Let $O$ be the origin. It is given that $OP$ increases at a rate not exceeding $32e^2$ units per minute. Someone claims that the area of $\triangle PQR$ increases at a rate lower than $2$ square units per minute when the $x$-coordinate of $P$ is $e$. Is the claim correct? Explain your answer. \\(4 marks) 
	\end{enumerate}

	\item \textbf{HKDSE Math M2 2018 Q10}
	\begin{enumerate}
		\item [(a)] 
		\begin{enumerate}
			\item [(i)]Prove that $\displaystyle \int \sin^4{x}\,dx = \frac{-\cos{x}\sin^3{x}}{4} + \frac{3}{4} \int \sin^2{x} \,dx$. 
			\item [(ii)] Evaluate $\displaystyle \int_{0}^{\pi} \sin^4{x}\,dx$.
		\end{enumerate}
		(5 marks)
		\item [(b)] 
		\begin{enumerate}
			\item [(i)]Let $f(x)$ be a continuous function such that $f(\beta - x)= f(x)$ for all real numbers $x$ , where $\beta$ is a constant. Prove that $\displaystyle\int_{0}^{\beta} x f(x) \,dx = \frac{\beta}{2} \int_{0}^{\beta } f(x) \,dx$.
			\item [(ii)] Evaluate $\displaystyle \int_{0}^{\pi} x\sin^4{x}\,dx$.
		\end{enumerate}
		(5 marks)
	\item [(c)]Consider the curve $G : y = \displaystyle \sqrt{x}\sin^2{x}$, where $\pi \leq x \leq 2\pi $. Let $R$ be the region bounded by $G$ and the $x$-axis. Find the volume of the solid of revolution generated by revolving $R$ about the $x$-axis. \\(3 marks) 
	\end{enumerate}

	\item \textbf{HKDSE Math M2 2018 Q11}
	\begin{enumerate}
		\item [(a)]Consider the system of linear equations in real variables $x$, $y$, $z$
		$$(E) : \left\{\begin{matrix}
		x&  +&ay&  +&4(a+1)z& = &18  \\
		2x& +&(a-1)y&  +&2(a-1)z& = & 20 \\
		x&  -&y&  -&12z& = & b \\
		\end{matrix}\right.\text{, where }a,b \in \mathbb{R}. $$
		\begin{enumerate}
			\item [(i)]Assume that $(E)$ has a unique solution.
			\begin{enumerate}
				\item [(1)]Find the range of values of $a$. 
				\item [(2)]Solve $(E)$. 
			\end{enumerate}			
			\item [(ii)]Assume that $a = 3 $ and $(E)$ is consistent.
			\begin{enumerate}
				\item [(1)]Find $b$. 
				\item [(2)]Solve $(E)$.
			\end{enumerate}
		\end{enumerate}
		(9 marks)
		\item [(b)]Consider the system of linear equations in real variables $x$, $y$, $z$
		$$(F) : \left\{\begin{matrix}
		x&  +&	3y&	+&	16z&	=& 18  \\
		x& 	+&	y&	+&	2z&		=& 20 \\
		x&  -&	y&  -&	12z&	=& s \\
		2x& -&	5y& -&	45z&	=& t \\
		\end{matrix}\right.\text{, where }s,t \in \mathbb{R}.$$
		Assume that $(F)$ is consistent. Find $s$ and $t$. \\(3 marks)
	\end{enumerate}

	\item \textbf{HKDSE Math M2 2018 Q2}\\
	The position vectors of the points $A, B, C$ and $D$ are  
	$4\textbf{i} -3 \textbf{j} + \textbf {k}$, 
	$-\textbf{i} +3 \textbf{j} -3 \textbf {k}$, 
	$7\textbf{i} - \textbf{j} +5 \textbf {k}$ and 
	$3\textbf{i} -2 \textbf{j} -5 \textbf {k}$  
	respectively. Denote the plane which contains $A, B$ and $C$ by $\Pi$. Let E be the projection of  $D$ on $\Pi$.
	\begin{enumerate}
		\item [(a)]Find
		\begin{enumerate}
			\item [(i)]$\overrightarrow{AB} \times \overrightarrow{AC}$,
			\item [(ii)]the volume of the tetrahedron $ABCD$,
			\item [(iii)]$\overrightarrow{DE}$.
		\end{enumerate}
		(5 marks)
		\item [(b)]Let $F$ be a point lying on $BC$ such that $DF$ is perpendicular to $BC$.
		\begin{enumerate}
			\item [(i)]Find $\overrightarrow{DF}$. 
			\item [(ii)]Is $\overrightarrow{BC} $ perpendicular to $\overrightarrow{EF}$ ? Explain your answer.
		\end{enumerate}
		(5 marks)
		\item[(c)]Find the angle between $\triangle BCD$ and $\Pi$. \\(3 marks)
	\end{enumerate}
\end{enumerate}
\end{document}