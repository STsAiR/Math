\documentclass[12pt]{article}
\usepackage{amsthm,amssymb,amsfonts,amsmath,amstext,systeme}

\marginparwidth 0pt
\oddsidemargin -1.2 truecm
\evensidemargin  0pt 
\marginparsep 0pt
\topmargin -2.2truecm
\linespread{1}
\textheight 25.8 truecm
\textwidth 18.5 truecm
\newenvironment{remark}{\noindent{\bf Remark }}{\vspace{0mm}}
\newenvironment{remarks}{\noindent{\bf Remarks }}{\vspace{0mm}}
\newenvironment{question}{\noindent{\bf Question }}{\vspace{0mm}}
\newenvironment{questions}{\noindent{\bf Questions }}{\vspace{0mm}}
\newenvironment{note}{\noindent{\bf Note }}{\vspace{0mm}}
\newenvironment{summary}{\noindent{\bf Summary }}{\vspace{0mm}}
\newenvironment{back}{\noindent{\bf Background}}{\vspace{0mm}}
\newenvironment{conclude}{\noindent{\bf Conclusion}}{\vspace{0mm}}
\newenvironment{concludes}{\noindent{\bf Conclusions}}{\vspace{0mm}}
\newenvironment{dill}{\noindent{\bf Description of Dill's model}}{\vspace{0mm}}
\newenvironment{maths}{\noindent{\bf Mathematics needed}}{\vspace{0mm}}
\newenvironment{inst}{\noindent{\bf Instructions}}{\vspace{0mm}}
\newenvironment{notes}{\noindent{\bf Notes }}{\vspace{0mm}}
\newenvironment{theorem}{\noindent{\bf Theorem }}{\vspace{0mm}}
\newenvironment{example}{\noindent{\bf Example }}{\vspace{0mm}}
\newenvironment{examples}{\noindent{\bf Examples }}{\vspace{0mm}}
\newenvironment{topics}{\noindent{\bf Topics}}{\vspace{0mm}}
\newenvironment{outcomes}{\noindent{\bf Expected Learning Outcomes}}{\vspace{0mm}}
\newenvironment{lemma}{\noindent{\bf Lemma }}{\vspace{0mm}}
\newenvironment{solution}{\noindent{\it Solution}}{\vspace{2mm}}
\newcommand{\ds}{\displaystyle}
\newcommand{\un}{\underline}
\newcommand{\bs}{\boldsymbol}

\begin{document}

\baselineskip 18 pt
\begin{center}
	{\large \bf HKDSE MATH M2 2020}\\
	\vspace{2 mm}

\end{center}
\vspace{0.05cm}

\begin{enumerate}
	\item \textbf{HKDSE Math M2 2020 Q1}
	\begin{enumerate}
		\item [(a)] Expand $(1-x)^4$.
		\item [(b)] Find the constant $k$ such that the coefficient of $x^2$ in the expansion of $(1+kx)^9(1-x)^4$ is $-3$.
	\end{enumerate}
	(4 marks)

	\item \textbf{HKDSE Math M2 2020 Q2}\\
	Define $\displaystyle f(x) = \frac{x}{\sqrt{2+x}}$, for all $x > -2$. Find $f'(2)$ from first principles. \\(4 marks)


	\item \textbf{HKDSE Math M2 2020 Q3}
	\begin{enumerate}
		\item [(a)] Let $x$ be an angle which is not a multiple of $30^\circ$. Prove that 
		\begin{enumerate}
			\item [(i)]$\tan{3x} = \displaystyle \frac{3\tan{x} - \tan^3{x}}{1-3\tan^2{x}}$, 
			\item [(ii)] $\tan{x} \tan(60^\circ - x)\tan(60^\circ + x) = \tan{3x}$. 
		\end{enumerate}
		\item [(b)] Using (a)(ii), prove that $\tan{55^\circ}\tan{65^\circ}\tan{75^\circ} = \tan{85^\circ}$.
	\end{enumerate}
	(6 marks)

	\item \textbf{HKDSE Math M2 2020 Q4}
	\begin{enumerate}
		\item [(a)]Find $\displaystyle \int \sin^2{\theta} \,d\theta$. 
		\item [(b)]Define $\displaystyle f(x) = 4x(1-x^2)^{\frac{1}{4}}$ for all $x \in [0,1]$. Denote the graph of $y = f(x) $ by $G$. Let $R$ be the region bounded by $G$ and the $x$-axis. Find the volume of the solid of revolution generated by revolving $R$ about the $x$-axis.
	\end{enumerate}
	(6 marks)

	\item \textbf{HKDSE Math M2 2020 Q5}
	\begin{enumerate}
		\item [(a)]Using mathematical induction, prove that $\displaystyle \sum_{k = 1}^{n} \frac{1}{k(k+1)(k+2)} = \frac{n(n+3)}{4(n+1)(n+2)}$ for all positive integers $n$.
		\item [(b)] Using (a), evaluate $\displaystyle \sum_{k = 4}^{123} \frac{50}{k(k+1)(k+2)}$.
	\end{enumerate}
	(7 marks)


	\item \textbf{HKDSE Math M2 2020 Q6}\\
	Consider the curve $C_1 : y = 2^{x-1}$, where $x>0$. Denote the region by $O$. Let $P(u,v)$ be a moving point on $C_1$ such that the area of the circle with $OP$ as a diameter increases at a constant rate of $5\pi$ square units per second.
	\begin{enumerate}
		\item[(a)]
		Define $S = u^2 + v^2$. Does $S$ increase at a constant rate? Explain your answer. 
		\item[(b)]
		Let $C_2$ be the curve $y = 2^x$, where $x>0$. The vertical line passing through $P$ cuts $C_2$ at the point $Q$. Find the rate of change of the area of $\triangle OPQ$ when $u=2$.
	\end{enumerate}
	(7 marks)

	\item \textbf{HKDSE Math M2 2020 Q7}\\
	Let $f(x)$ be a continuous function defined on $\mathbb{R}$. Denote the curve $y = f(x)$ by $\Gamma$. It is given that $\Gamma $ passes through the point $(1,2)$ and $f'(x) = -2x+8$ for all $x \in \mathbb{R}$. 
	\begin{enumerate}
		\item [(a)]Find the equation of $\Gamma$. 
		\item [(b)]Let L be a tangent to $\Gamma$ such that $L$ passes through the point $(5,14)$ and the slope of $L$ is negative. Denote the point of contact of $\Gamma$ and $L$ by $P$. Find 
		\begin{enumerate}
			\item [(i)]the coordinates of $P$, 
			\item [(ii)]the equation of the normal to $\Gamma $ at $P$.
		\end{enumerate}
	\end{enumerate}
	(8 marks)

	\item \textbf{HKDSE Math M2 2020 Q8}\\
	Define $P = 
				\begin{pmatrix}
				-5&-2\\
				15&6\\
				\end{pmatrix}$ and $Q = 
				\begin{pmatrix}
				1&0\\
				0&0\\
				\end{pmatrix}$. Let $M = 
				\begin{pmatrix}
				1&a\\
				b&c\\
				\end{pmatrix}$  such that $|M| = 1$ and $PM = MQ$, where $a$, $b$ and $c$ are real numbers.
	\begin{enumerate}
		\item [(a)] Find $a,b$ and $c$. 
		\item [(b)] Define $R = 
				\begin{pmatrix}
				6&2\\
				-15&-5\\
				\end{pmatrix}$. 
		\begin{enumerate}
			\item [(i)]Evaluate $M^{-1}RM$. 
			\item [(ii)]Using the result of (b)(i), prove that $(\alpha P + \beta R)^{99} = \alpha^{99}P+\beta ^{99}R$ for any real numbers $\alpha$ and $\beta$. 
		\end{enumerate} 
	\end{enumerate}
	(8 marks)

	\item \textbf{HKDSE Math M2 2020 Q9}\\
	Let $\displaystyle f(x) = \frac{(x+4)^3}{(x-4)^2}$ for all real numbers $x \neq 4$. Denote the graph of $y = f(x)$ by $H$.
	\begin{enumerate}
		\item [(a)] Find the asymptote(s) of $H$. \\(3 marks)
		\item [(b)] Find $f''(x)$. \\(2 marks)
		\item [(c)] Someone claims that there are two turning points of $H$. Do you agree? Explain your answer. \\(2 marks)
		\item [(d)] Find the point(s) of inflexion of $H$. \\(2 marks)
		\item [(e)] Find the area of the region bounded by $H$, the $x$-axis and the $y$-axis. \\ (3 marks)
	\end{enumerate}


	\item \textbf{HKDSE Math M2 2020 Q10}
	\begin{enumerate}
		\item [(a)]Using integration by substitution, prove that $\displaystyle \int_{\frac{\pi}{12}}^{\frac{\pi}{6}}  \ln{\left(\sin{\left(\frac{\pi}{4} - x\right)}\right)}\,dx = \int_{\frac{\pi}{12}}^{\frac{\pi}{6}}  \ln{(\sin{x})}\,dx$. \\(3 marks)
		\item [(b)] Using (a), evaluate $\displaystyle \int_{\frac{\pi}{12}}^{\frac{\pi}{6}}  \ln{(\cot{x} - 1)}\,dx$. \\(3 marks)
		\item [(c)] 
		\begin{enumerate}
			\item [(i)] Using $\cot{(A-B)} = \displaystyle \frac{\cot{A}\cot{B}+1}{\cot{B} - \cot{A}}$, or otherwise, prove that $\displaystyle \cot{\frac{\pi}{12}} = 2 + \sqrt{3}$. 
			\item [(ii)] Using integration by parts, prove that $\displaystyle\int_{ \frac{\pi}{12}}^{\frac{\pi}{6}}  \frac{x\csc^2{x}}{\cot{x} -1}\,dx = \frac{\pi}{8} \ln{(2 + \sqrt{3})}$。
		\end{enumerate}
	(7 marks)
	\end{enumerate}

	\item \textbf{HKDSE Math M2 2020 Q11}
	\begin{enumerate}
		\item [(a)] Consider the system of linear equations in real variables $x$, $y$, $z$
		$$(E):\left\{\begin{matrix}
		x&  -& y& -&2z& = & 1  \\
		x&  -&2y& +&hz& = & k \\
		4x& +&hy& -&7z& = & 7 \\
		\end{matrix}\right. \text{, where }h,k \in \mathbb{R}.$$
		\begin{enumerate}
			\item [(i)]Assume that $(E)$ has a unique solution.
			\begin{enumerate}
				\item [(1)]Prove that $h \neq -3$. 
				\item [(2)]Solve $(E)$.
			\end{enumerate}
			\item [(ii)]Assume that $h = -3$ and $(E)$ is consistent.
			\begin{enumerate}
				\item [(1)]Prove that $k = -2$. 
				\item [(2)]Solve $(E)$. 
			\end{enumerate}
		\end{enumerate}
		(9 marks)
		\item[(b)]Consider the system of linear equations in real variables $x$, $y$, $z$
			$$(F):\left\{\begin{matrix}
			x&  -&y&  -&2z& = &1  \\
			x&  -&2y&  +&hz& = & -2 \\
			4x&  +&hy&  -&7z& = & 7 \\
			\end{matrix}\right. \text{, where }h \in \mathbb{R}.$$
			Someone claims that there are at least two values of $h$ such that $(F)$ has a real solution $(x,y,z)$ satisfying $3x^2+4y^2-7z^2=1$. Do you agree? Explain your answer. \\(4 marks)

	\end{enumerate}


	\item \textbf{HKDSE Math M2 2020 Q12}\\
	Let $\overrightarrow{OP} = \textbf{i} + \textbf{j}+ 4\textbf {k}$ and $\overrightarrow{OQ} = 5\textbf{i} -7 \textbf{j}- 4\textbf {k}$, where $O$ is the origin. $R$ is a point lying on $PQ$ such that $PR:RQ = 1:3$. 
	\begin{enumerate}
		\item [(a)]Find $\overrightarrow{OP} \times \overrightarrow{OR}$. \\(2 marks)
		\item [(b)]Define $\overrightarrow{OS} = \overrightarrow{OP} + \overrightarrow{OR}$. Find the area of the quadrilateral $OPSR$. \\(2 marks)
		\item [(c)]Let $N$ be a point such that $\overrightarrow{ON} = \lambda(\overrightarrow{OP}\times \overrightarrow{OR})$, where $\lambda$ is a real number.
		\begin{enumerate}
			\item [(i)]Is $\overrightarrow{NR}$ perpendicular to $\overrightarrow{PQ}$? Explain your answer.
			\item [(ii)]Let $\mu$ be a real number such that $\overrightarrow{NQ}$ is parallel to $11\textbf{i} + \mu\textbf{j}-10\textbf {k}$. 
			\begin{enumerate}
				\item [(1)]Find $\lambda$ and $\mu$. 
				\item [(2)]Denote the angle between $\triangle OPQ$ and $\triangle NPQ$ by $\theta$. Find $\tan{\theta}$.
			\end{enumerate}
		\end{enumerate}
		(8 marks)
	\end{enumerate}
\end{enumerate}
\end{document}