\documentclass[12pt]{article}
\usepackage{amsthm,amssymb,amsfonts,amsmath,amstext,systeme}

\marginparwidth 0pt
\oddsidemargin -1.2 truecm
\evensidemargin  0pt 
\marginparsep 0pt
\topmargin -2.2truecm
\linespread{1}
\textheight 25.8 truecm
\textwidth 18.5 truecm
\newenvironment{remark}{\noindent{\bf Remark }}{\vspace{0mm}}
\newenvironment{remarks}{\noindent{\bf Remarks }}{\vspace{0mm}}
\newenvironment{question}{\noindent{\bf Question }}{\vspace{0mm}}
\newenvironment{questions}{\noindent{\bf Questions }}{\vspace{0mm}}
\newenvironment{note}{\noindent{\bf Note }}{\vspace{0mm}}
\newenvironment{summary}{\noindent{\bf Summary }}{\vspace{0mm}}
\newenvironment{back}{\noindent{\bf Background}}{\vspace{0mm}}
\newenvironment{conclude}{\noindent{\bf Conclusion}}{\vspace{0mm}}
\newenvironment{concludes}{\noindent{\bf Conclusions}}{\vspace{0mm}}
\newenvironment{dill}{\noindent{\bf Description of Dill's model}}{\vspace{0mm}}
\newenvironment{maths}{\noindent{\bf Mathematics needed}}{\vspace{0mm}}
\newenvironment{inst}{\noindent{\bf Instructions}}{\vspace{0mm}}
\newenvironment{notes}{\noindent{\bf Notes }}{\vspace{0mm}}
\newenvironment{theorem}{\noindent{\bf Theorem }}{\vspace{0mm}}
\newenvironment{example}{\noindent{\bf Example }}{\vspace{0mm}}
\newenvironment{examples}{\noindent{\bf Examples }}{\vspace{0mm}}
\newenvironment{topics}{\noindent{\bf Topics}}{\vspace{0mm}}
\newenvironment{outcomes}{\noindent{\bf Expected Learning Outcomes}}{\vspace{0mm}}
\newenvironment{lemma}{\noindent{\bf Lemma }}{\vspace{0mm}}
\newenvironment{solution}{\noindent{\it Solution}}{\vspace{2mm}}
\newcommand{\ds}{\displaystyle}
\newcommand{\un}{\underline}
\newcommand{\bs}{\boldsymbol}

\begin{document}

\baselineskip 18 pt
\begin{center}
	{\large \bf HKDSE MATH M2 2019}\\
	\vspace{2 mm}

\end{center}
\vspace{0.05cm}

\begin{enumerate}
	\item \textbf{HKDSE Math M2 2019 Q1}\\
	Let $\displaystyle f(x) = \frac{10x}{7+3x^2}$. Prove that $f(1+h) - f(1) = \displaystyle\frac{4h-3h^2}{10+6h+3h^2}$. Hence, find  $f'(1)$ from first principles. \\(4 marks)


	\item \textbf{HKDSE Math M2 2019 Q2}\\
	Let $P(x) = \begin{vmatrix}
		x+\lambda & 1 & 2  \\ 
		0 & (x+\lambda)^2 & 3 \\ 
		4 & 5 & (x+\lambda)^3  \notag
		\end{vmatrix}$, where $\lambda \in \mathbb{R}$. It is given that the coefficient of $x^3$ in the expansion of $P(x)$ is 160. Find
	\begin{enumerate}
		\item $\lambda$,
		\item $P'(0)$. 
	\end{enumerate}
	(5 marks)




	\item \textbf{HKDSE Math M2 2019 Q3}\\
	A researcher performs an experiment to study the rate of change of the volume of liquid $X$ in a vessel. The experiment lasts for $24$ hours. At the start  of the experiment, the vessel contains $580 $cm$^3$ of liquid $X$. The researcher finds that during the experiment, $\displaystyle\frac{dV}{dt} = -2t$, where $V $cm$^3$ is the volume of liquid $X$ in the vessel and $t$ is the number of hours elapsed since the start of the experiment.
	\begin{enumerate}
		\item [(a)] The researcher claims that the vessel contains some liquid $X$ at the end of the experiment. Is the claim correct? Explain your answer.
		\item [(b)] It is given that $V = h^2+24h$, where $h $ cm is the depth of liquid $X$ in the vessel. Find the value of $\displaystyle\frac{dh}{dt}$ when $t = 18$.
	\end{enumerate}
	(6 marks)

	\item \textbf{HKDSE Math M2 2019 Q4}\\
	Define $\displaystyle g(x) = \frac{\ln{x}}{\sqrt{x}}$ for all $x \in (0,99)$. Denote the graph of $y = g(x) $ by $G$. 
	\begin{enumerate}
		\item [(a)]Prove that $G$ has only one maximum point. 
		\item [(b)]Let $R$ be the region bounded by $G$, the $x$-axis and the vertical line passing through the maximum point of $G$. Find the volume of the solid of revolution generated by revolving $R$ about the $x$-axis.
	\end{enumerate}
	(6 marks)

	\item\textbf{HKDSE Math M2 2019 Q5}
	\begin{enumerate}
		\item [(a)] Using mathematical induction, prove that $\displaystyle \sum_{k = n}^{2n} \frac{1}{k(k+1)} = \frac{(n+1)}{n(2n+1)}$ for all positive integers $n$.
		\item [(b)] Using (a), evaluate $\displaystyle \sum_{k = 50}^{200} \frac{1}{k(k+1)}$.
	\end{enumerate}
	(7 marks)


	\item \textbf{HKDSE Math M2 2019 Q6}\\
	Consider the system of linear equations in real variables $x$, $y$, $z$
		$$(E):\left\{\begin{matrix}
		 x&  -&2y&  -&2z& = &\beta  \\
		5x&  +&\alpha y&  +&\alpha z& = & 5\beta \\
		7x&  +&(\alpha - 3)y&  +&(2\alpha+1)z& = & 8\beta \\
		\end{matrix}\right.\text{, where }\alpha, \beta \in \mathbb{R}.$$
		\begin{enumerate}
			\item [(a)]Assume that $(E)$ has a unique solution.
			\begin{enumerate}
				\item [(i)]Find the range of values of $\alpha$. 
				\item [(ii)]Express $y$ in terms of $\alpha$ and $\beta$. 
			\end{enumerate}
			\item [(b)]Assume that $\alpha = -4$ . If  $(E)$ is inconsistent, find the range of values of $\beta$.
		\end{enumerate}
	(7 marks)

	\item \textbf{HKDSE Math M2 2019 Q7}
	\begin{enumerate}
		\item [(a)]Using integration by parts, find $\displaystyle\int e^x\sin{\pi x}\, dx$. 
		\item [(b)]Using integration by substitution, evaluate $\displaystyle\int_{0}^{3} e^{3-x}\sin{\pi x} \,dx$.
	\end{enumerate}
	(7 marks)

	\item \textbf{HKDSE Math M2 2019 Q8}\\
	Let $h(x)$ be a continuous function defined on $\mathbb{R}^+$, where $\mathbb{R}^+$ is the set of positive real numbers. It is given that $\displaystyle h'(x) = \frac{2x^2 -7x+8}{x}$ for all $x > 0$. 
	\begin{enumerate}
		\item [(a)] Is $h(x)$ an increasing function? Explain your answer. 
		\item [(b)] Denote the curve $y = h(x)$ by $H$. It is given that $H$ passes through the point $(1,3)$. Find  
		\begin{enumerate}
			\item [(i)]the equation of $H$,
			\item [(ii)]the point(s) of inflexion of $H$.
		\end{enumerate} 
	\end{enumerate}
	(8 marks)

	\item \textbf{HKDSE Math M2 2019 Q9}\\
	Consider the curve $\Gamma : y = \displaystyle\frac{1}{3}\sqrt{12-x^2}$, where $0<x<2\sqrt{3}$. Denote the tangent of $\Gamma$ at $x = 3$ by $L$.  
	\begin{enumerate}
		\item [(a)]Find the equation of $L$. \\(3 marks)
		\item [(b)]Let $C$ be the curve $y = \sqrt{4-x^2}$, where $0<x<2$. It is given that $L$ is a tangent to $C$. Find 
		\begin{enumerate}
			\item [(i)]the point(s) of contact of $L$ and $C$;  
			\item [(ii)]the point(s) of intersection of $C$ and $\Gamma$;  
			\item [(iii)]the area of region bounded by $L$, $C$ and $\Gamma$.
		\end{enumerate}
		(9 marks)
	\end{enumerate}

	\item \textbf{HKDSE Math M2 2019 Q10}
	\begin{enumerate}
		\item [(a)] Let $0 \leq x \leq \displaystyle\frac{\pi}{4}$. Prove that $\displaystyle\frac{1}{2+\cos{2x}} = \frac{\sec^2{x}}{2+\sec^2{x}}$. \\(1 mark) 
		\item [(b)] Evaluate $\displaystyle \int_{0}^{ \frac{\pi}{4}} \frac{1}{2+\cos{2x}}\,dx$. \\(3 marks)
		\item [(c)] Let $f(x)$ be a continuous function defined on $\mathbb{R}$ such that $f(-x) = -f(x)$ for all $x \in \mathbb{R}$. \\Prove that $\displaystyle\int_{-a}^{a} f(x)\ln{(1+e^x)}\,dx = \int_{0}^{a} xf(x)\,dx$ for any $a \in \mathbb{R}  $. \\(4 marks)
		\item [(d)] Evaluate $\displaystyle \int_{ \frac{-\pi}{4}}^{ \frac{\pi}{4}}  \frac{\sin{2x}}{(2+\cos{2x})^2}\ln(1 + e^x)\,dx$. \\(5 marks)
	\end{enumerate}


	\item \textbf{HKDSE Math M2 2019 Q11}\\
	Let $M = \begin{pmatrix}
		2 &7 \\
		-1&-6\\
	\end{pmatrix}$. Denote the $2 \times2$ identity matrix by $I$. 
	\begin{enumerate}
		\item[(a)]Find a pair of real numbers $a$ and $b$ such that $M^2 = aM + bI$. \\(3 marks) 
		\item[(b)]
		Prove that $6M^n = (1-(-5)^n)M+(5+(-5)^n)I)$ for all positive integers $n$. \\(4 marks)
		\item[(c)]Does there exist a pair of $2\times2$ real matrices $A$ and $B$ such that $(M^n)^{-1} = A +\displaystyle \frac{1}{(-5)^n}B$ for all positive integers $n$? Explain your answer. \\(5 marks)
	\end{enumerate}


	\item \textbf{HKDSE Math M2 2019 Q12}\\
	Let $\overrightarrow{OA} = \textbf{i} -4 \textbf{j}+ 2\textbf {k}$, $\overrightarrow{OB} = -5\textbf{i} -4 \textbf{j} +8\textbf {k}$ and $\overrightarrow{OC} = -5\textbf{i} -12 \textbf{j} +t\textbf {k}$, where $O$ is the origin and $t$ is a constant. It is given that $|\overrightarrow{AC}| = |\overrightarrow{BC}|$. 
	\begin{enumerate}
		\item [(a)]Find $t$. \\(3 marks)
		\item [(b)]Find $\overrightarrow{AB} \times \overrightarrow{AC}$. \\(2 marks)
		\item [(c)]Find the volume of the pyramid $OABC$. \\(2 marks)
		\item [(d)]Denote the plane which contains $A$, $B$ and $C$ by $\Pi$. It is given that $P$, $Q$ and $R$ are points lying on $\Pi$ such that $\overrightarrow{OP} = p\textbf{i}$, $\overrightarrow{OQ} = q\textbf{j}$ and $\overrightarrow{OQ} = r\textbf{k}$. Let $D$ be the projection of $O$ on $\Pi$.
		\begin{enumerate}
			\item [(i)]Prove that $pqr \neq 0$. 
			\item [(ii)]Find $\overrightarrow{OD}$. 
			\item [(ii)]Let $E$ be a point such that $\overrightarrow{OE} = \displaystyle\frac{1}{p}\textbf{i}+\frac{1}{q}\textbf{j}+\frac{1}{r}\textbf{k}$. Describe the geometric relationship between $D$, $E$ and $O$. Explain your answer.
		\end{enumerate}
		(6 marks)
	\end{enumerate}
\end{enumerate}
\end{document}