\documentclass[12pt]{article}
\usepackage{amsthm,amssymb,amsfonts,amsmath,amstext,systeme}

\marginparwidth 0pt
\oddsidemargin -1.2 truecm
\evensidemargin  0pt 
\marginparsep 0pt
\topmargin -2.2truecm
\linespread{1}
\textheight 25.8 truecm
\textwidth 18.5 truecm
\newenvironment{remark}{\noindent{\bf Remark }}{\vspace{0mm}}
\newenvironment{remarks}{\noindent{\bf Remarks }}{\vspace{0mm}}
\newenvironment{question}{\noindent{\bf Question }}{\vspace{0mm}}
\newenvironment{questions}{\noindent{\bf Questions }}{\vspace{0mm}}
\newenvironment{note}{\noindent{\bf Note }}{\vspace{0mm}}
\newenvironment{summary}{\noindent{\bf Summary }}{\vspace{0mm}}
\newenvironment{back}{\noindent{\bf Background}}{\vspace{0mm}}
\newenvironment{conclude}{\noindent{\bf Conclusion}}{\vspace{0mm}}
\newenvironment{concludes}{\noindent{\bf Conclusions}}{\vspace{0mm}}
\newenvironment{dill}{\noindent{\bf Description of Dill's model}}{\vspace{0mm}}
\newenvironment{maths}{\noindent{\bf Mathematics needed}}{\vspace{0mm}}
\newenvironment{inst}{\noindent{\bf Instructions}}{\vspace{0mm}}
\newenvironment{notes}{\noindent{\bf Notes }}{\vspace{0mm}}
\newenvironment{theorem}{\noindent{\bf Theorem }}{\vspace{0mm}}
\newenvironment{example}{\noindent{\bf Example }}{\vspace{0mm}}
\newenvironment{examples}{\noindent{\bf Examples }}{\vspace{0mm}}
\newenvironment{topics}{\noindent{\bf Topics}}{\vspace{0mm}}
\newenvironment{outcomes}{\noindent{\bf Expected Learning Outcomes}}{\vspace{0mm}}
\newenvironment{lemma}{\noindent{\bf Lemma }}{\vspace{0mm}}
\newenvironment{solution}{\noindent{\it Solution}}{\vspace{2mm}}
\newcommand{\ds}{\displaystyle}
\newcommand{\un}{\underline}
\newcommand{\bs}{\boldsymbol}

\begin{document}

\baselineskip 18 pt
\begin{center}
	{\large \bf HKDSE MATH M2 2017}\\
	\vspace{2 mm}

\end{center}
\vspace{0.05cm}

\begin{enumerate}
	\item \textbf{HKDSE Math M2 2017 Q1}\\
	Let $\displaystyle \frac{d}{d\theta} \sec{6\theta}$ from first principles. \\(5 marks)


	\item \textbf{HKDSE Math M2 2017 Q2}\\
	Let $\displaystyle(1+ax)^8 = \sum_{k = 0}^{8} \lambda _{k} x^{k}$ and $\displaystyle(b+x)^9 = \sum_{k = 0}^{9} \mu _{k} x^{k}$, where $a$ and $b$ are constants. It is given that $\lambda_2 : \mu_7 = 7:4$ and $\lambda_1 + \mu_8  + 6 = 0$. Find $a$.  \\(5 marks)


	\item \textbf{HKDSE Math M2 2017 Q3}\\
	$P$ is a point lying on $AB$ such that $AP : PB = 3:2$. Let $\overrightarrow{OA} = \textbf{a}$ and $\overrightarrow{OB} = \textbf{b}$, where $O$ is the origin.
	\begin{enumerate}
		\item [(a)] Express $\overrightarrow{OP} $ in therms of $ \textbf{a}$ and $ \textbf{b}$.
		\item [(b)] It is given that $|\textbf{a}| = 45$, $|\textbf{b}| = 20$ and $\cos{\angle{AOB}} = \displaystyle\frac{1}{4}$. Find
		\begin{enumerate}
			\item [(i)] $\textbf{a}   \cdot  \textbf{b} $, 
			\item [(ii)] $|\overrightarrow{OP}| $. 
		\end{enumerate}
	\end{enumerate}
	(5 marks)

	\item \textbf{HKDSE Math M2 2017 Q4}	
	\begin{enumerate}
		\item [(a)]Using integration by parts, find $\displaystyle\int x^2 e^{-x} \,dx$. 
		\item [(b)]Find the area of the region bounded by the graph of $y = x^2 e^{-x}$, the $x$-axis and the straight line $x = 6$.
	\end{enumerate}
	(6 marks)

	\item \textbf{HKDSE Math M2 2017 Q5}\\
	Consider the following system of linear equations in real variables $x$, $y$, $z$
		$$(E) : \left\{\begin{matrix}
		x&  +&2y&  -&z& = &11  \\
		3x& +&8y&  -&11z& = & 49 \\
		2x& +&3y&  +&hz& = & k \\
		\end{matrix}\right.\text{, where } h,k \in \mathbb{R}.$$ 
	\begin{enumerate}
		\item [(a)] Assume that $(E)$ has a unique solution.
		\begin{enumerate}
			\item [(i)]Find the range of values of $h$.
			\item [(ii)]Express $z$ in terms of $h$ and $k$.
		\end{enumerate}
		\item [(b)]Assume that $(E)$ has infinitely many solutions. Solve $(E)$.
	\end{enumerate}
	(6 marks)


	\item \textbf{HKDSE Math M2 2017 Q6}\\
	A container in the form of an inverted right circular cone is held vertically. The height and the base radius of the container are 20 cm and 15 cm respectively. Water is now poured into the container.
	\begin{enumerate}
		\item [(a)]Let $A$ cm$^2$ be the wet curve surface area of the container and $h$ cm be the depth of water in the container. Prove that $\displaystyle A = \frac{15}{16}\pi h^2$. 
		\item [(b)]The depth of water in the container increases at a constant rate of $\displaystyle\frac{3}{\pi}$ cm/s. Find the rate of change of the wet curved surface area of the container when the volume of water in the container is $96 \pi$ cm$^3$.
	\end{enumerate}
	(7 marks)

	\item\textbf{HKDSE Math M2 2017 Q7}
	\begin{enumerate}
		\item[(a)]Prove that $\sin{3x} = 3\sin{x}  - 4\sin^3{x}$. 
		\item[(b)]Let $\displaystyle\frac{\pi}{4} < x < \frac{\pi}{2}$.
		\begin{enumerate}
			\item [(i)]Prove that $\displaystyle\frac{\sin{3\left(x - \displaystyle\frac{\pi}{4}\right)}}{\sin{\left(x - \displaystyle\frac{\pi}{4}\right)}} = \frac{\cos{3x} + \sin{3x}}{\cos{x} - \sin{x}}$
			\item [(ii)]Solve the equation $\displaystyle\frac{\cos{3x} + \sin{3x}}{\cos{x} - \sin{x}} = 2$.
		\end{enumerate}
	\end{enumerate}
	(8 marks)

	\item \textbf{HKDSE Math M2 2017 Q8}\\
	Let $f(x) $ be a continuous function defined on  $\mathbb{R} ^+$, where $\mathbb{R} ^+$ is the set of positive real numbers. Denote the curve $y = f(x)$ by  $\Gamma$. It is given that $\Gamma$ passes through the point $P(e^3 , 7)$ and $f'(x) = \displaystyle\frac{1}{x} \ln{x^2}$ for all $x>0$. Find 
	\begin{enumerate}
		\item [(a)] the equation of the tangent to $\Gamma$ at $P$,
		\item [(b)] the equation of $\Gamma$,
		\item [(c)] the point(s) of inflexion of $\Gamma$.
	\end{enumerate}
	(8 marks)

	\item \textbf{HKDSE Math M2 2017 Q9}\\
	Define $f(x) = \displaystyle\frac{x^2 - 5x}{x + 4}$ for all $x \neq -4$. Denote the graph of $y = f(x)$ by $G$. 
	\begin{enumerate}
		\item [(a)]Find the asymptote(s) of $G$.  \\(3 marks)
		\item [(b)]Find $f'(x)$. \\(2 marks) 
		\item [(c)]Find the maximum point(s) and the minimum point(s) of $G$. \\(4 marks) 
		\item [(d)]Let $R$ be the region bounded by $G$ and the $x$-axis. Find the volume of the solid of revolution generated by revolving $R$ about the $x$-axis. \\(4 marks)
	\end{enumerate}


	\item \textbf{HKDSE Math M2 2017 Q10}\\
	$ABC$ is a triangle. $D$ is the mid-point of $AC$. $E$ is a point lying on $BC$ such that $BE : EC = 1 : r$. $AB$ produced and $DE$ produced meet at the point $F$. It is given that $DE : EF = 1 : 10$. Let 
		$\overrightarrow{OA} = 2\textbf{i} +3 \textbf{j} -2\textbf {k}$ , 
		$\overrightarrow{OB} = 4\textbf{i} +4 \textbf{j} - \textbf {k}$ , 
		$\overrightarrow{OC} = 8\textbf{i} -3 \textbf{j} -2\textbf {k}$ , where $O$ is the origin.
	\begin{enumerate}
		\item [(a)]By expressing $\overrightarrow{AE}$ and $\overrightarrow{AF}$ in terms of $r$, find $r$.\\(4 marks)
		\item [(b)]
		\begin{enumerate}
			\item [(i)]Find $\overrightarrow{AD} \cdot \overrightarrow{DE}$. 
			\item [(ii)]Are $B, D, C$ and $F$ concyclic? Explain your answer.
		\end{enumerate}
		(5 marks)
		\item[(c)]Let $\overrightarrow{OP} = 3\textbf{i} +10 \textbf{j} -4\textbf {k}$. Denote the circumcentre of $\triangle BCF $ by $ Q$. Find the volume of the tetrahedron $ABPQ$. \\(3 marks)
	\end{enumerate}

	\item \textbf{HKDSE Math M2 2017 Q11}
	\begin{enumerate}
		\item [(a)]Using $\displaystyle\tan^{-1}{\sqrt{2}} - \tan^{-1}{\left(\frac{\sqrt{2}}{2}\right)} = \tan^{-1}{\left(\frac{\sqrt{2}}{4}\right)}$, evaluate $\displaystyle\int_{0}^{1} \frac{1}{x^2+2x+3}\,dx $. \\(3 marks)
		\item [(b)]
		\begin{enumerate}
			\item [(i)]Let $0 \leq \theta \leq \displaystyle\frac{\pi}{4}$ . Prove that $\displaystyle\frac{2\tan{\theta}}{1 + \tan^2{\theta}} = \sin{2\theta}$ and $\displaystyle\frac{1-\tan^2{\theta}}{1 + \tan^2{\theta}} = \cos{2\theta}$.
			\item [(ii)]Using the substitution $t = \tan{\theta}$, evaluate $\displaystyle\int_{0}^{\frac{\pi}{4}} \frac{1}{\sin{2\theta} + \cos{2\theta} + 2} \,d\theta$.
		\end{enumerate}
		(5 marks)
		\item [(c)]Prove that $\displaystyle\int_{ 0}^{\frac{\pi}{4}} \frac{\sin{2\theta}+1 }{\sin{2\theta} + \cos{2\theta} + 2} \,d\theta = \displaystyle\int_{ 0}^{\frac{\pi}{4}} \frac{\cos{2\theta}+1}{\sin{2\theta} + \cos{2\theta} + 2} \,d\theta$. \\(2 marks)
		\item [(d)]Evaluate $\displaystyle\int_{ 0}^{\frac{\pi}{4}} \frac{8\sin{2\theta} + 9}{\sin{2\theta} + \cos{2\theta} + 2} \,d\theta$. \\(3 marks)
	\end{enumerate}



	\item \textbf{HKDSE Math M2 2017 Q12}\\
	Let $A = \begin{pmatrix}
		3 & 1\\
		0 & 3\\
	\end{pmatrix}$. Denote the $2 \times 2$ identity matrix by $I$.
	\begin{enumerate}
		\item [(a)]Using mathematical induction, prove that $A^n = 3^nI + 3^{n-1}n\begin{pmatrix}
			0&1\\0&0\\
			\end{pmatrix}$ for all positive integers $n$. \\(4 marks)

		\item [(b)]Let $B = \begin{pmatrix}
			5 & 1\\
			-4 & 1\\
			\end{pmatrix}$. 
		\begin{enumerate}
			\item [(i)]Define $P = \begin{pmatrix}
			-1 & 0\\
			2 & -1\\
			\end{pmatrix}$. Evaluate $P^{-1}BP$.  
			\item [(ii)]Prove that $B^n = 3^nI + 3^{n-1}n\begin{pmatrix}
			2&1\\-4&-2\\
			\end{pmatrix}$ for any positive integer $n$.
			\item [(iii)]Does there exist a positive integer $m$ such that $|A^m - B^m| = 4m^2$ ? Explain your answer.
		\end{enumerate}
		(8 marks)
	\end{enumerate}	
\end{enumerate}
\end{document}