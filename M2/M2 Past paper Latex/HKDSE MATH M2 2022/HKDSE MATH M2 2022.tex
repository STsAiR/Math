\documentclass[12pt]{article}
\usepackage{amsthm,amssymb,amsfonts,amsmath,amstext,systeme}

\marginparwidth 0pt
\oddsidemargin -1.2 truecm
\evensidemargin  0pt 
\marginparsep 0pt
\topmargin -2.2truecm
\linespread{1}
\textheight 25.8 truecm
\textwidth 18.5 truecm
\newenvironment{remark}{\noindent{\bf Remark }}{\vspace{0mm}}
\newenvironment{remarks}{\noindent{\bf Remarks }}{\vspace{0mm}}
\newenvironment{question}{\noindent{\bf Question }}{\vspace{0mm}}
\newenvironment{questions}{\noindent{\bf Questions }}{\vspace{0mm}}
\newenvironment{note}{\noindent{\bf Note }}{\vspace{0mm}}
\newenvironment{summary}{\noindent{\bf Summary }}{\vspace{0mm}}
\newenvironment{back}{\noindent{\bf Background}}{\vspace{0mm}}
\newenvironment{conclude}{\noindent{\bf Conclusion}}{\vspace{0mm}}
\newenvironment{concludes}{\noindent{\bf Conclusions}}{\vspace{0mm}}
\newenvironment{dill}{\noindent{\bf Description of Dill's model}}{\vspace{0mm}}
\newenvironment{maths}{\noindent{\bf Mathematics needed}}{\vspace{0mm}}
\newenvironment{inst}{\noindent{\bf Instructions}}{\vspace{0mm}}
\newenvironment{notes}{\noindent{\bf Notes }}{\vspace{0mm}}
\newenvironment{theorem}{\noindent{\bf Theorem }}{\vspace{0mm}}
\newenvironment{example}{\noindent{\bf Example }}{\vspace{0mm}}
\newenvironment{examples}{\noindent{\bf Examples }}{\vspace{0mm}}
\newenvironment{topics}{\noindent{\bf Topics}}{\vspace{0mm}}
\newenvironment{outcomes}{\noindent{\bf Expected Learning Outcomes}}{\vspace{0mm}}
\newenvironment{lemma}{\noindent{\bf Lemma }}{\vspace{0mm}}
\newenvironment{solution}{\noindent{\it Solution}}{\vspace{2mm}}
\newcommand{\ds}{\displaystyle}
\newcommand{\un}{\underline}
\newcommand{\bs}{\boldsymbol}

\begin{document}

\baselineskip 18 pt
\begin{center}
	{\large \bf HKDSE MATH M2 2022}\\
	\vspace{2 mm}

\end{center}
\vspace{0.05cm}

\begin{enumerate}
	\item \textbf{HKDSE Math M2 2022 Q1}\\
	Let $\displaystyle g(x) = \frac{1}{\sqrt{5x+4}}$, where $x > 0$. Prove that $\displaystyle g(1+h)-g(1) = \frac{-5h}{3\sqrt{5h+9}(3+\sqrt{5h+9})}$. Hence, find $g'(1)$ from first principles. \\
	(4 marks)

	\item \textbf{HKDSE Math M2 2022 Q2}\\
	Let $\displaystyle \frac{\pi}{4} < \theta < \frac{\pi}{2}$.
	\begin{enumerate}
		\item [(a)] Prove that $\displaystyle \frac{\tan{\theta}}{1-\cot{\theta}} + \frac{\cot{\theta}}{1-\tan{\theta}} = 1+\sec{\theta}\csc{\theta}$.
		\item [(b)] Solve the equation $\displaystyle \frac{\tan{\theta}}{1-\cot{\theta}} + \frac{\cot{\theta}}{1-\tan{\theta}} = 5$.
	\end{enumerate}
	(5 marks)

	\item \textbf{HKDSE Math M2 2022 Q3}
	\begin{enumerate}
		\item [(a)]Using mathematical induction, prove that $\displaystyle \sum_{k = 1}^{2n} (-1)^k k^2 = n(2n+1)$ for all positive integers $n$.
		\item [(b)] Using (a), evaluate $\displaystyle \sum_{k = 11}^{100} (-1)^k k^2$. 
	\end{enumerate}
	(7 marks)


	\item \textbf{HKDSE Math M2 2022 Q4}\\
	Let $y = (7x - 2x^2)e^{-x}$.
	\begin{enumerate}
		\item[(a)]
		Find $\displaystyle \frac{dy}{dx}$ and $\displaystyle \frac{d^2y}{dx^2}$. 
		\item[(b)]
		Someone claims that there are two points of inflexion of the graph of $y = (7x - 2x^2)e^{-x}$. Do you agree? Explain your answer. 
	\end{enumerate}
	(6 marks)

	\item \textbf{HKDSE Math M2 2022 Q5}\\
	Let $n$ be an integer greater than 1. Define $\displaystyle (a+x)^n  = \sum_{k = 0}^{n} \mu_k x^k$, where $a$ is a constant. It is given that $\mu_2 = -10$.
	\begin{enumerate}
		\item [(a)] Explain why $a$ is a negative number and $n$ is an odd number. 
		\item [(b)] Let $\displaystyle (bx-1)^n  = \sum_{k = 0}^{n} \lambda_k x^k$, where $b$ is a constant. If $\lambda_0 = \mu_0$ and $\lambda_1 = 2\mu_1$, find $a$, $b$ and $n$.
	\end{enumerate}
	(6 marks)

	\item \textbf{HKDSE Math M2 2022 Q6}
	\begin{enumerate}
		\item [(a)]Using integration by substitution, prove that $\displaystyle \int \frac{1}{x^2+2x+5}\,dx = \frac{1}{2} \tan^{-1} \left(\frac{x+1}{2}\right) +$ constant.
		\item [(b)]At any point $(x,y)$ on the curve $G$, the slope of the tangent to $G$ is $\displaystyle \frac{2x+1}{x^2+2x+5}$. Given that $G$ passes through the point $(-3, \ln{2})$, does $G$ pass through the point $\displaystyle \left(-1, \frac{-\pi}{8}\right)$ ? Explain your answer.
	\end{enumerate}
	(7 marks)

	\item \textbf{HKDSE Math M2 2022 Q7}\\
	Consider the curve $\Gamma : y = \ln{(x+2)}$, where $x > 0$. Let $P$ be a moving point on $\Gamma$ with $h$ as its $x$-coordinate. Denote the tangent to $\Gamma$ at $P$ by $L$ and the area of the region bounded by $\Gamma$, $L$ and the $y$-axis by $A$ square units.
	\begin{enumerate}
		\item [(a)]Prove that $\displaystyle A = \frac{h^2+4h}{2h+4} - 2\ln{(h+2)} + 2\ln{2}$.
		\item [(b)]If $h = 3^{-t}$, where $t$ is the time measured in seconds, find the rate of change of $A$ when $t = 1$.
	\end{enumerate}
	(8 marks)

	\item \textbf{HKDSE Math M2 2022 Q8}\\
	Consider the system of linear equations in real variables $x$, $y$ and $z$
		$$(E) : \left\{\begin{matrix}
		ax&  +&2y&  -& z& = &4k  \\
		-x&  +&ay&  +&2z& = &4   \\
		2x&  -& y&  +&az& = &k^2 \\
		\end{matrix}\right. , \text{where } a,k \in \mathbb{R}$$
	\begin{enumerate}
		\item [(a)] Assume that $(E)$ has a unique solution. Express $y$ in terms of $a$ and $k$.
		\item [(b)] Assume that $(E)$ has infinitely many solutions. Solve $(E)$. 
	\end{enumerate}
	(7 marks)

	\item \textbf{HKDSE Math M2 2022 Q9}\\
	Let $\displaystyle f(x) = \frac{x^2 +3x}{x-1}$, where $x \neq 1$. Denote the graph of $y = f(x)$ by $H$.
	\begin{enumerate}
		\item [(a)] Find the asymptote(s) of $H$. \\(3 marks)
		\item [(b)] Find the maximum point(s) and minimum point(s) of $H$. \\(4 marks)
		\item [(c)] Sketch $H$. \\(3 marks)
		\item [(d)] Let $R$ be the region bounded by $H$ and the straight line $y = 10$. Find the volume of the solid of revolution generated by revolving $R$ about the straight line $y = 10$. \\(3 marks)
	\end{enumerate}

	\item \textbf{HKDSE Math M2 2022 Q10}\\
	Let $g(x) = \cos^2{x}\cos{2x}$.
	\begin{enumerate}
		\item [(a)] Prove that $\displaystyle \int g(x) \,dx = \frac{\sin{2x}\cos^2{x}}{2} + \frac{1}{2} \int \sin^2{2x}\,dx$. \\(2 marks)
		\item [(b)] Evaluate $\displaystyle \int _{0}^{\pi} g(x) \,dx$. \\(2 marks)
		\item [(c)] Using integration by substitution, evaluate $\displaystyle \int _{0}^{\pi} xg(x) \,dx$. \\(4 marks)
		\item [(d)] Evaluate $\displaystyle \int _{-\pi}^{2\pi} xg(x) \,dx$. \\(4 marks)
	\end{enumerate}

	\item \textbf{HKDSE Math M2 2022 Q11}
	\begin{enumerate}
		\item [(a)] Let $n$ be a positive integer. Denote the $2\times2$ identify matrix by $I$. 
		\begin{enumerate}
			\item [(i)] Let $A$ be a $2\times2$ matrix. Simplify $(I - A)(I + A + A^2 + \cdots + A^n)$.
			\item [(ii)]Let $A = 
				\begin{pmatrix}
				\cos{\theta}&-\sin{\theta}\\
				\sin{\theta}&\cos{\theta}\\
				\end{pmatrix} \text{, 
				where } \theta \text{ is not a multiple of } 2\pi$.\\
				It is given that $A^n = 
				\begin{pmatrix}
				\cos{n\theta}&-\sin{n\theta}\\
				\sin{n\theta}&\cos{n\theta}\\
				\end{pmatrix}$. 
			\begin{enumerate}
				\item [(1)] Prove that $\displaystyle(I - A) ^{-1} = \frac{1}{2\sin{\displaystyle\frac{\theta}{2}}} 
					\begin{pmatrix}
						\sin{\displaystyle\frac{\theta}{2}}&-\cos{\displaystyle\frac{\theta}{2}}\\
						\cos{\displaystyle\frac{\theta}{2}}&\sin{\displaystyle\frac{\theta}{2}}\\
					\end{pmatrix}$
				\item[(2)] Using the result of (a)(i) and (a)(ii)(1),\\prove that $I+A+A^2+\cdots+A^n = \displaystyle\frac{\sin{\displaystyle\frac{(n+1)\theta}{2}}}{\sin{\displaystyle\frac{\theta}{2}}}
					\begin{pmatrix}
						\cos{\displaystyle\frac{n\theta}{2}}&-\sin{\displaystyle\frac{n\theta}{2}}\\
						\sin{\displaystyle\frac{n\theta}{2}}&\cos{\displaystyle\frac{n\theta}{2}}\\
					\end{pmatrix}$ . 
			\end{enumerate}
		\end{enumerate}
		(7 marks)
		\item[(b)]Using (a)(ii), evaluate
		\begin{enumerate}
			\item [(i)]$\cos{\displaystyle\frac{5\pi}{18}}+\cos{\displaystyle\frac{5\pi}{9}}+\cos{\displaystyle\frac{5\pi}{6}}+\cdots+\cos{25\pi}$ ; 
			\item [(ii)]$\cos^2{\displaystyle\frac{\pi}{7}}+\cos^2{\displaystyle\frac{2\pi}{7}}+\cos^2{\displaystyle\frac{3\pi}{7}}+\cdots+\cos^2{7\pi}$. 
		\end{enumerate}
	(6 marks)
	\end{enumerate}


	\item \textbf{HKDSE Math M2 2022 Q12}\\
	Consider $\triangle ABC$. Denote the origin by $O$. 
	\begin{enumerate}
		\item [(a)]Let $D$ be a point lying on $BC$ such that $AD$ is the angle bisector of $\angle BAC$. Define $BC = a$, $AC = b$  and  $AB = c$. 
		\begin{enumerate}
			\item [(i)]Using the fact that $BD:DC = c:b$, prove that $\overrightarrow{AD} = -\overrightarrow{OA}+\displaystyle\frac{b}{b+c}\overrightarrow{OB} + \displaystyle\frac{c}{b+c}\overrightarrow{OC}$. 
			\item [(ii)]Let $E$ be a point lying on $AC$ such that $BE$ is the angle bisector of $\angle ABC$. \\
				Define $\overrightarrow{OJ} = \displaystyle\frac{a}{a+b+c}\overrightarrow{OA}+\displaystyle\frac{b}{a+b+c}\overrightarrow{OB}+\displaystyle\frac{c}{a+b+c}\overrightarrow{OC}$. \\
				Prove that $J$ lies on $AD$. Hence, deduce that $AD$ and $BE$ intersect at $J$.
		\end{enumerate}
		(7 marks)
		\item [(b)]Suppose that $\overrightarrow{OA} = 35\textbf{i} +9 \textbf{j}+ \textbf {k}$, $\overrightarrow{OB} = 40\textbf{i} -3 \textbf{j}+ \textbf {k}$ and $\overrightarrow{OC} =  -3 \textbf{j}+ \textbf {k}$. Let $I$ be the incentre of $\triangle ABC$. 
		\begin{enumerate}
			\item [(i)] Find $\overrightarrow{OI}$. 
			\item [(ii)] By considering $\overrightarrow{AI} \times \overrightarrow{AB}$, find the radius of the inscribed circle of $\triangle ABC$. 
		\end{enumerate}
		(5 marks)
	\end{enumerate}
\end{enumerate}
\end{document}