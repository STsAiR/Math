\documentclass[12pt]{article}
\usepackage{amsthm,amssymb,amsfonts,amsmath,amstext,systeme}

\marginparwidth 0pt
\oddsidemargin -1.2 truecm
\evensidemargin  0pt 
\marginparsep 0pt
\topmargin -2.2truecm
\linespread{1}
\textheight 25.8 truecm
\textwidth 18.5 truecm
\newenvironment{remark}{\noindent{\bf Remark }}{\vspace{0mm}}
\newenvironment{remarks}{\noindent{\bf Remarks }}{\vspace{0mm}}
\newenvironment{question}{\noindent{\bf Question }}{\vspace{0mm}}
\newenvironment{questions}{\noindent{\bf Questions }}{\vspace{0mm}}
\newenvironment{note}{\noindent{\bf Note }}{\vspace{0mm}}
\newenvironment{summary}{\noindent{\bf Summary }}{\vspace{0mm}}
\newenvironment{back}{\noindent{\bf Background}}{\vspace{0mm}}
\newenvironment{conclude}{\noindent{\bf Conclusion}}{\vspace{0mm}}
\newenvironment{concludes}{\noindent{\bf Conclusions}}{\vspace{0mm}}
\newenvironment{dill}{\noindent{\bf Description of Dill's model}}{\vspace{0mm}}
\newenvironment{maths}{\noindent{\bf Mathematics needed}}{\vspace{0mm}}
\newenvironment{inst}{\noindent{\bf Instructions}}{\vspace{0mm}}
\newenvironment{notes}{\noindent{\bf Notes }}{\vspace{0mm}}
\newenvironment{theorem}{\noindent{\bf Theorem }}{\vspace{0mm}}
\newenvironment{example}{\noindent{\bf Example }}{\vspace{0mm}}
\newenvironment{examples}{\noindent{\bf Examples }}{\vspace{0mm}}
\newenvironment{topics}{\noindent{\bf Topics}}{\vspace{0mm}}
\newenvironment{outcomes}{\noindent{\bf Expected Learning Outcomes}}{\vspace{0mm}}
\newenvironment{lemma}{\noindent{\bf Lemma }}{\vspace{0mm}}
\newenvironment{solution}{\noindent{\it Solution}}{\vspace{2mm}}
\newcommand{\ds}{\displaystyle}
\newcommand{\un}{\underline}
\newcommand{\bs}{\boldsymbol}

\begin{document}

\baselineskip 18 pt
\begin{center}
	{\large \bf MI Note}\\
	\vspace{2 mm}

\end{center}
\vspace{0.05cm}

\section{\textbf{Mathematical induction}}


\raggedright {\bf Stage 0} Let $P(n)$ be the desired statement.\\
{\bf Stage 1} (Initial Kick)$$\text{For }n = 1, \text{ show that }P(1) \text{ is true.}$$\\
{\bf Stage 2} (Induction Hypothesis)$$\text{Assume }P(k)\text{ is true for some positive integer }k.$$\\
{\bf Stage 3} (Inductive Step - From $n$ to $n+1$)$$\text{For }n = k+1,\text{ show that }P(k+1)\text{ is true.}$$\\
{\bf Stage 4} (End of Story)$$\text{Therefore }P(n)\text{is true for ALL positive integers }n\text{ by M.I.}$$\\



\section{\textbf{Exercise}}


Prove, by mathematical induction, that $$1+3+5+\ldots+(2n-1)=n^2$$ for all positive integer $n$.\\.\\
Let $P(n)$ be the desired statement.\\
For $n=1$, we have LHS $ = 1 = 1^2 =$ RHS. Hence $P(1)$ is true.\\
Assume $P(k)$ is true for some positive $k$.\\
For $n = k+1$
\begin{align*}
	1 + 3 + \ldots + (2k-1) + (2k+1) &= k^2 + (2k+1)\\
&=(k+1)^2
\end{align*}
Hence $P(k+1)$ is true assuming $P(k)$ is true.\\
$\therefore P(n)$ is true for all positive integer $n\geq 1$ by induction.

\begin{enumerate}
	\item \textbf{HKCEE A.Maths 1994 Past Paper II Q5}\\
	Prove by induction that $$\displaystyle \frac{1}{2}+\frac{3}{2^2}+\frac{5}{2^3}+\ldots+\frac{2n-1}{2^n}=3-\frac{2n+3}{2^n}$$ for all positive integers $n$.
	\newpage	
	\item \textbf{HKCEE A.Maths 1994 Past Paper II Q5}\\
	Prove by induction that $$\displaystyle \frac{1}{2}+\frac{3}{2^2}+\frac{5}{2^3}+\ldots+\frac{2n-1}{2^n}=3-\frac{2n+3}{2^n}$$ for all positive integers $n$.

	\item \textbf{HKCEE A.Maths 2005 Past Paper Q8}\\
	Prove by induction that $$\displaystyle \frac{1 \ldots 2}{2 \ldots 3}+\frac{2 \ldots 2^2}{3 \ldots 4}+\frac{3 \ldots 2^3}{4 \ldots 5}+\ldots+\frac{n\ldots 2^n}{(n+1)(n+2)}=\frac{2^{n+1}-(n+2)}{n+2}$$ for all positive integers $n$.

	\item \textbf{HKCEE A.Maths 2007 Past Paper Q5}\\
	Let $a \neq 0$ and $a \neq 1$. Prove by mathematical induction, that $$\displaystyle \frac{1}{a-1} - \frac{1}{a} - \frac{1}{a^2} - \cdots - \frac{1}{a^n} = \frac{1}{a^{n+1}-a^n}$$ for all positive integers $n$.

\end{enumerate}


\end{document}