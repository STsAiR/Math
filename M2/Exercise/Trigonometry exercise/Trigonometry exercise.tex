\documentclass[12pt]{article}
\usepackage{amsthm,amssymb,amsfonts,amsmath,amstext,systeme}

\marginparwidth 0pt
\oddsidemargin -1.2 truecm
\evensidemargin  0pt 
\marginparsep 0pt
\topmargin -2.2truecm
\linespread{1}
\textheight 25.8 truecm
\textwidth 18.5 truecm
\newenvironment{remark}{\noindent{\bf Remark }}{\vspace{0mm}}
\newenvironment{remarks}{\noindent{\bf Remarks }}{\vspace{0mm}}
\newenvironment{question}{\noindent{\bf Question }}{\vspace{0mm}}
\newenvironment{questions}{\noindent{\bf Questions }}{\vspace{0mm}}
\newenvironment{note}{\noindent{\bf Note }}{\vspace{0mm}}
\newenvironment{summary}{\noindent{\bf Summary }}{\vspace{0mm}}
\newenvironment{back}{\noindent{\bf Background}}{\vspace{0mm}}
\newenvironment{conclude}{\noindent{\bf Conclusion}}{\vspace{0mm}}
\newenvironment{concludes}{\noindent{\bf Conclusions}}{\vspace{0mm}}
\newenvironment{dill}{\noindent{\bf Description of Dill's model}}{\vspace{0mm}}
\newenvironment{maths}{\noindent{\bf Mathematics needed}}{\vspace{0mm}}
\newenvironment{inst}{\noindent{\bf Instructions}}{\vspace{0mm}}
\newenvironment{notes}{\noindent{\bf Notes }}{\vspace{0mm}}
\newenvironment{theorem}{\noindent{\bf Theorem }}{\vspace{0mm}}
\newenvironment{example}{\noindent{\bf Example }}{\vspace{0mm}}
\newenvironment{examples}{\noindent{\bf Examples }}{\vspace{0mm}}
\newenvironment{topics}{\noindent{\bf Topics}}{\vspace{0mm}}
\newenvironment{outcomes}{\noindent{\bf Expected Learning Outcomes}}{\vspace{0mm}}
\newenvironment{lemma}{\noindent{\bf Lemma }}{\vspace{0mm}}
\newenvironment{solution}{\noindent{\it Solution}}{\vspace{2mm}}
\newcommand{\ds}{\displaystyle}
\newcommand{\un}{\underline}
\newcommand{\bs}{\boldsymbol}

\begin{document}

\baselineskip 18 pt
\begin{center}
	{\large \bf Trigonometry Exercise}\\
	\vspace{2 mm}

\end{center}
\vspace{0.05cm}

\begin{enumerate}
	\item Let $x$ be a real number such that $\sec{x} - \tan{x} = 2$. Find the value of $\sec{x} + \tan{x}$.\\Ans : 1/2\newpage
	\item Find the maximum value of $y = 5 + \displaystyle\frac{4}{2\sec^2{x}-1}$.\\Ans : 9\newpage
	\item Find the minimum value of $y = 3 - \displaystyle\frac{1}{4\csc^2{2x}-3}$.\\Ans : 2\newpage
	\item 
	\begin{enumerate}
		\item Find the maximum and minimum value of $1- \displaystyle\frac{3y^2}{4}$ where $0 \leq y \leq 1$.
		\item 
		\begin{enumerate}
			\item Express $\sin^4{x} + \cos^4{x}$ in terms of $\sin{2x}$.
			\item Hence express $\sin^6{x} + \cos^6{x}$ in terms of $\sin{2x}$.
		\end{enumerate}
		\item Using (a) and (b), or otherwise, find the maximum and minimum value of $\sin^6{x} + \cos^6{x}$.
	\end{enumerate}
	Ans : (a) Min = 1/4, Max = 1 (b)(i) $1-\displaystyle\frac{\sin^2{2x}}{2}$ (ii) $1-\displaystyle\frac{3\sin^2{2x}}{4}$ (c) Min = 1/4, Max = 1\newpage
	\item Prove that $$\displaystyle\frac{\sin{3x} - \sin{x}}{\cos{x} - \cos{3x}} = \cot{2x}$$ \newpage
	\item Let $y = \displaystyle\cos{\frac{\pi}{7}}\cos{\frac{2\pi}{7}}\cos{\frac{4\pi}{7}}$, by considering $\displaystyle y\sin{\frac{\pi}{7}}$, find the value of $y$.\\Ans : $-1/8$\newpage
	\item Let $y = \displaystyle\cos{\frac{\pi}{15}}\cos{\frac{2\pi}{15}}\cos{\frac{4\pi}{15}}\cos{\frac{8\pi}{15}}$, by considering $\displaystyle y\sin{\frac{\pi}{15}}$, find the value of $y$.\\Ans : $-1/16$\newpage
	\item Let $y = \displaystyle\cos{\frac{\pi}{7}} + \cos{\frac{3\pi}{7}} + \cos{\frac{5\pi}{7}}$.
	\begin{enumerate}
		\item Prove that $\displaystyle 2y\sin{\frac{\pi}{7}} = \sin{\frac{6\pi}{7}}$.
		\item Using (a), find the value of $y$.
		\item Using (b), find the value of $\displaystyle\cos{\frac{2\pi}{7}} + \cos{\frac{4\pi}{7}} + \cos{\frac{6\pi}{7}}$. [Hint: $\cos{(\pi - x)} = -\cos{x}$]
	\end{enumerate}
	Ans : (b) $y = 1/2$ (c) $-1/2$\newpage
	\item Solve the equation $$\sin{2x} + \sin{4x} = \cos{x}$$ for $0 \leq x \leq \pi$. \\Ans : $\pi/18$, $5\pi/18$, $\pi/2$, $13\pi/18$, $17\pi/18$\newpage
	\item 
	\textbf{HKALE Pure Math 2007 Paper 2 Q11(b)(i)(ii)(Modified)}
	\begin{enumerate}
		\item Prove that $\displaystyle\tan{\frac{3\pi}{8}} = 1 + \sqrt{2}$. 
		\item Using (a), prove that $\displaystyle \tan{\frac{\pi}{24}} = \frac{1 + \sqrt{2} - \sqrt{3}}{1 + \sqrt{3} + \sqrt{6}}$. 
	\end{enumerate}
	\newpage
	\item Prove by mathematical induction that $$\sin{x} + \sin{3x} + \cdots + \sin{(2n-1)x} = \displaystyle\frac{\sin^2{nx}}{\sin{x}}$$ where $\sin{x} \neq 0$, for all positive integers $n$.\newpage
\end{enumerate}

\end{document}