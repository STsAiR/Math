\documentclass[12pt]{article}
\usepackage{amsthm,amssymb,amsfonts,amsmath,amstext,systeme}

\marginparwidth 0pt
\oddsidemargin -1.2 truecm
\evensidemargin  0pt 
\marginparsep 0pt
\topmargin -2.2truecm
\linespread{1}
\textheight 25.8 truecm
\textwidth 18.5 truecm
\newenvironment{remark}{\noindent{\bf Remark }}{\vspace{0mm}}
\newenvironment{remarks}{\noindent{\bf Remarks }}{\vspace{0mm}}
\newenvironment{question}{\noindent{\bf Question }}{\vspace{0mm}}
\newenvironment{questions}{\noindent{\bf Questions }}{\vspace{0mm}}
\newenvironment{note}{\noindent{\bf Note }}{\vspace{0mm}}
\newenvironment{summary}{\noindent{\bf Summary }}{\vspace{0mm}}
\newenvironment{back}{\noindent{\bf Background}}{\vspace{0mm}}
\newenvironment{conclude}{\noindent{\bf Conclusion}}{\vspace{0mm}}
\newenvironment{concludes}{\noindent{\bf Conclusions}}{\vspace{0mm}}
\newenvironment{dill}{\noindent{\bf Description of Dill's model}}{\vspace{0mm}}
\newenvironment{maths}{\noindent{\bf Mathematics needed}}{\vspace{0mm}}
\newenvironment{inst}{\noindent{\bf Instructions}}{\vspace{0mm}}
\newenvironment{notes}{\noindent{\bf Notes }}{\vspace{0mm}}
\newenvironment{theorem}{\noindent{\bf Theorem }}{\vspace{0mm}}
\newenvironment{example}{\noindent{\bf Example }}{\vspace{0mm}}
\newenvironment{examples}{\noindent{\bf Examples }}{\vspace{0mm}}
\newenvironment{topics}{\noindent{\bf Topics}}{\vspace{0mm}}
\newenvironment{outcomes}{\noindent{\bf Expected Learning Outcomes}}{\vspace{0mm}}
\newenvironment{lemma}{\noindent{\bf Lemma }}{\vspace{0mm}}
\newenvironment{solution}{\noindent{\it Solution}}{\vspace{2mm}}
\newcommand{\ds}{\displaystyle}
\newcommand{\un}{\underline}
\newcommand{\bs}{\boldsymbol}

\begin{document}

\baselineskip 18 pt
\begin{center}
	{\large \bf Differentiation Note}\\
	\vspace{2 mm}
\end{center}
\vspace{0.05cm}
\section{Definition of Differentiation}
The slope two points in a function:
$$\displaystyle\lim_{x \to a} \frac{f(x) - f(a)}{x-a} $$
Differentiation(First principle):
$$\displaystyle\frac{dy}{dx} = \frac{d}{dx}f(x)= f'(x)=\lim_{h \to 0}  \frac{f(x+h) - f(x)}{h} $$
$$\displaystyle\frac{dy}{dx}\Bigr\rvert_{x = a} = f'(a)=\lim_{h \to 0}  \frac{f(a+h) - f(a)}{h} $$
\newpage
\section{Exercise}
\begin{enumerate}
	\item \textbf{HKDSE Math M2 Sample Paper Q1}\\
	Find $\displaystyle\frac{d}{dx}(\sqrt{2x})$ from the first principles. \\(4 marks)
	\item \textbf{HKDSE Math M2 Practice Paper Q6}\\
	Find $\displaystyle\frac{d}{dx}\left(\frac{1}{x}\right)$ from the first principles. \\(4 marks)	
	\item \textbf{HKDSE Math M2 2012 Q1}\\
	Let $f(x) = e^{2x}$. Find $f'(0)$ from first principles. \\(3 marks)	
	\item \textbf{HKDSE Math M2 2013 Q1}\\
	Find $\displaystyle\frac{d}{dx} (\sin{2x})$ from first principles. \\(4 marks)	
	\item \textbf{HKDSE Math M2 2014 Q2 (a)}\\
	Consider the curve $C : y = x^3-3x$. 
	\begin{enumerate}
		\item [(a)]Find $\displaystyle\frac{dy}{dx}$ from first principles. 
	\end{enumerate}
	\item \textbf{HKDSE Math M2 2015 Q1}\\
	Find $\displaystyle \frac{d}{dx} (x^5+4)$ from first principles. \\(4 marks)
	\item \textbf{HKDSE Math M2 2016 Q2}\\
	Prove that $\displaystyle\frac{1}{\sqrt{x}} - \frac{1}{\sqrt{x+h}} = \frac{h}{(x+h)\sqrt{x} + x\sqrt{x+h}}$. Hence, find $\displaystyle \frac{d}{dx} \sqrt{\displaystyle\frac{3}{x}}$ from first principles. \\(5 marks)
	\item \textbf{HKDSE Math M2 2017 Q1}\\
	Let $\displaystyle \frac{d}{d\theta} \sec{6\theta}$ from first principles. \\(5 marks)	
	\item \textbf{HKDSE Math M2 2018 Q1}\\
	Let $\displaystyle f(x) = (x^2-1)e^x$.  Express $f(1+h)$ in terms of $h$. Hence, find  $f'(1)$ from first principles. \\(4 marks)	
	\item \textbf{HKDSE Math M2 2019 Q1}\\
	Let $\displaystyle f(x) = \frac{10x}{7+3x^2}$. Prove that $f(1+h) - f(1) = \displaystyle\frac{4h-3h^2}{10+6h+3h^2}$. Hence, find  $f'(1)$ from first principles. \\(4 marks)
	\item \textbf{HKDSE Math M2 2020 Q2}\\
	Define $\displaystyle f(x) = \frac{x}{\sqrt{2+x}}$, for all $x > -2$. Find $f'(2)$ from first principles. \\(4 marks)
	\item \textbf{HKDSE Math M2 2021 Q1}\\
	Let $\displaystyle f(x) = \frac{1}{3x^{2}+4}$. Find $f'(x)$ from first principles. \\(4 marks)
	\item \textbf{HKDSE Math M2 2022 Q1}\\
	Let $\displaystyle g(x) = \frac{1}{\sqrt{5x+4}}$, where $x > 0$. Prove that $\displaystyle g(1+h)-g(1) = \frac{-5h}{3\sqrt{5h+9}(3+\sqrt{5h+9})}$. Hence, find $g'(1)$ from first principles. \\
	(4 marks)
\end{enumerate}




\section{Rules of Differentiation}
\begin{enumerate}
	\item $\displaystyle\frac{d}{dx}C = 0$, where $C$ is a constant
	\item $\displaystyle\frac{d}{dx}x^n = nx^{n-1}$, where n is a constant
	\item $\displaystyle\frac{d}{dx}a^x = a^x\ln{a}$ where $a$ is a positive constant and $a \neq 1$
	\item $\displaystyle\frac{d}{dx}e^{x} = e^{x}$
	\item $\displaystyle\frac{d}{dx}\ln{x} = \frac{1}{x}$
	\item $\displaystyle\frac{d}{dx}\sin{x} = \cos{x}$
	\item $\displaystyle\frac{d}{dx}\cos{x} = -\sin{x}$
	\item $\displaystyle\frac{d}{dx}\tan{x} = \sec^2{x}$
\end{enumerate}
\section{Quiz}
\begin{enumerate}
	\item [(a)] Given $f(x) = x^2$, find $f'(x)$
	\item [(b)] Prove (a)
\end{enumerate}

\newpage
\section{Sum and Difference Rule}
$$\displaystyle\frac{d}{dx}[f(x)\pm g(x)] = \frac{d}{dx}f(x) \pm \frac{d}{dx}g(x)$$
\section{Product Rule}
$$\displaystyle\frac{d}{dx}[f(x)\times g(x)] = g(x)\frac{d}{dx}f(x) + f(x)\frac{d}{dx}g(x)$$
\section{Quotient Rule}
$$\displaystyle\frac{d}{dx}\frac{f(x)}{g(x)} = \frac{g(x)\frac{d}{dx}f(x) - f(x)\frac{d}{dx}g(x)}{[g(x)]^2}$$
\section{Chain Rule}
$$\displaystyle\frac{d}{dx}f(g(x)) = \frac{d\,f(g(x))}{d(g(x))}\,\cdot\,\frac{d(g(x))}{dx} = f'(g(x))\,\cdot\, g'(x)$$
$$\displaystyle\frac{dy}{dx} = \frac{dy}{du}\,\cdot\,\frac{du}{dx} $$

\end{document}