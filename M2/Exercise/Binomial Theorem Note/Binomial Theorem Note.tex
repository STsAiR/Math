\documentclass[12pt]{article}
\usepackage{amsthm,amssymb,amsfonts,amsmath,amstext,systeme}

\marginparwidth 0pt
\oddsidemargin -1.2 truecm
\evensidemargin  0pt 
\marginparsep 0pt
\topmargin -2.2truecm
\linespread{1}
\textheight 25.8 truecm
\textwidth 18.5 truecm
\newenvironment{remark}{\noindent{\bf Remark }}{\vspace{0mm}}
\newenvironment{remarks}{\noindent{\bf Remarks }}{\vspace{0mm}}
\newenvironment{question}{\noindent{\bf Question }}{\vspace{0mm}}
\newenvironment{questions}{\noindent{\bf Questions }}{\vspace{0mm}}
\newenvironment{note}{\noindent{\bf Note }}{\vspace{0mm}}
\newenvironment{summary}{\noindent{\bf Summary }}{\vspace{0mm}}
\newenvironment{back}{\noindent{\bf Background}}{\vspace{0mm}}
\newenvironment{conclude}{\noindent{\bf Conclusion}}{\vspace{0mm}}
\newenvironment{concludes}{\noindent{\bf Conclusions}}{\vspace{0mm}}
\newenvironment{dill}{\noindent{\bf Description of Dill's model}}{\vspace{0mm}}
\newenvironment{maths}{\noindent{\bf Mathematics needed}}{\vspace{0mm}}
\newenvironment{inst}{\noindent{\bf Instructions}}{\vspace{0mm}}
\newenvironment{notes}{\noindent{\bf Notes }}{\vspace{0mm}}
\newenvironment{theorem}{\noindent{\bf Theorem }}{\vspace{0mm}}
\newenvironment{example}{\noindent{\bf Example }}{\vspace{0mm}}
\newenvironment{examples}{\noindent{\bf Examples }}{\vspace{0mm}}
\newenvironment{topics}{\noindent{\bf Topics}}{\vspace{0mm}}
\newenvironment{outcomes}{\noindent{\bf Expected Learning Outcomes}}{\vspace{0mm}}
\newenvironment{lemma}{\noindent{\bf Lemma }}{\vspace{0mm}}
\newenvironment{solution}{\noindent{\it Solution}}{\vspace{2mm}}
\newcommand{\ds}{\displaystyle}
\newcommand{\un}{\underline}
\newcommand{\bs}{\boldsymbol}

\begin{document}

\baselineskip 18 pt
\begin{center}
	{\large \bf Binomial Theorem Note}\\
	\vspace{2 mm}

\end{center}
\vspace{0.05cm}

\section{\textbf{Binomial Theorem}}
Concept.$$(x+y)^n=\binom{n}{0}x^n+\binom{n}{1}x^{n-1}y+\binom{n}{2}x^{n-2}y^2+\cdots+\binom{n}{n-1}xy^{n-1}+\binom{n}{n}y^n.$$
Where,$$(x+y)^n=\sum_{r=0}^n\binom{n}{r}x^ry^{n-r}=\sum_{r=0}^n\binom{n}{r}x^{n-r}y^r$$ 
$$\binom{n}{r}=\dfrac{n!}{r!(n-r)!}$$




\section{\textbf{Exercise}}




\begin{enumerate}
	\item Warm up:
	\begin{enumerate}
		\item Expand $(1 + x)^7$ in ascending powers of $x$ up to $x^4$.
		\item Expand $(2x^2+1)^5$ in ascending powers of $x$ up to $x^7$.
		\item Expand $(9-x)^6$ in descending powers of $x$ up to $x^4$.
		\item Find the coefficient of $x^3y^5$ in the expansion $(2x-y)^8$. 
	\end{enumerate}
	Ans:
	\begin{enumerate}
		\item $1 + 7x + 21x^2 + 35x^3 + 35x^4 + \cdots$
		\item $1 + 10x^2 + 40x^4 + 80x^6 + \cdots$
		\item $x^6-54x^5+1215x^4+ \cdots$
		\item $-448$
	\end{enumerate}
	\item 
	\textbf{HKCEE A. Maths 2004 Q3}
	\begin{enumerate}
		\item Expand $(1+2x)^6$ in ascending powers of $x$ up to the terms $x^3$. 
		\item Find the constant term in the expansion of $\left(\displaystyle1 - \frac{1}{x} + \frac{1}{x^2}\right)(1+2x)^6$. 
	\end{enumerate}
	Ans:
	\begin{enumerate}
		\item $a + 12x + 60x^2 + 160x^3 + \cdots$
		\item 49
	\end{enumerate}
	\item 
	\textbf{HKCEE A.Maths 2009 Q11}\\
	In the binomial expansion of $\left(\displaystyle x^2 + \frac{1}{x}\right)^{20}$, find
	\begin{enumerate}
		\item the coefficient of $x^{16} $.
		\item the constant term.
	\end{enumerate}
	Ans:
	\begin{enumerate}
		\item 125970
		\item 0
	\end{enumerate}
	\item 
	\textbf{HKDSE M2 2012 Q2}\\
	It is given that 
	$$(1+ax)^n = 1 + 6x + 16x^2 + \text{terms involving higher powers of } x$$
	where $n$ is a positive integer and $a$ is a constant. Find the values of $a$ and $n$.\\
	Ans: $a = 2/3 , n = 9$
	\item 
	\textbf{HKDSE M2 2013 Q2}\\	
	Suppose the coefficient of $x$ and $x^2$ in the expansion of $(1+ax)^n$ are $-20$ and 180 respectively. Find the values of $a$ and $n$.\\
	Ans: $a = -2 , n = 10$
	\item 
	\textbf{HKDSE M2 2014 Q1}\\
	In the expansion of $(1-4x)^2(1+x)^n$, the coefficient of $x$ is 1.
	\begin{enumerate}
		\item [(a)]Find the value of $n$. 
		\item [(b)]Find the coefficient of $x^2$.
	\end{enumerate}
	Ans:
	\begin{enumerate}
		\item $n = 9$
		\item Coefficient of $x^2 = -20$
	\end{enumerate}
	\item Given $$1-x+x^2-x^3+\cdots+x^{16}-x^{17} = a_0 + a_1y + a_2y^2 + \cdots + a_{16}y^{16} + a_{17}y^{17}$$ where $y = x+1$ and $a_k's$ are constants. Find $a_2$.\\
	Ans: $a_2 = \displaystyle\sum_{k = 2}^{17} k(k-1) = 816$
	\item Let $(1+x)^{100} = a_0 + a_1x+a_2x^2+\cdots +a_{99}x^{99} + a_{100}x^{100}$.90
	\begin{enumerate}
		\item Expand $(1-x)^{100}$ in terms of $a_0 ,a_1 \cdots , a_{100}$.
		\item Using (a), or otherwise, show that 
		\begin{enumerate}
			\item $a_0 - a_1 + a_2 - a_3 + \cdots - a_{99} + a_{100} = 0$
			\item $a_0 + a_2 + a_4 + \cdots + a_{98} + a_{100} = 2^{99}$
		\end{enumerate}
		\item Using (a), or otherwise, show that $$a_1 - 2a_2x + 3a_3x^2 - \cdots + 99a_{99}x^{98} - 100a_{100}x^{99} = 100(1-x)^{99}$$
	\end{enumerate}
	Ans: 
	\begin{enumerate}
		\item $a_0-a_1x+a_2x^2+\cdots -a_{99}x^{99} + a_{100}x^{100}$
	\end{enumerate}
\end{enumerate}
\end{document}